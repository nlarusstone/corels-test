\section{Learning Optimal Rule Lists}
\label{sec:framework}

\begin{arxiv}
In this section, we present our framework for learning certifiably optimal rule lists.
%
First, we define our setting and useful notation~(\S\ref{sec:setup})
and then the objective function we seek to minimize~(\S\ref{sec:objective}).
%
Next, we describe the principal structure of our optimization algorithm~(\S\ref{sec:optimization}), which depends on a hierarchically
structured objective lower bound~(\S\ref{sec:hierarchical}).
%
We then derive a series of additional bounds that we incorporate into our
algorithm, because they enable aggressive pruning of our state space.
\end{arxiv}

\subsection{Notation}
\label{sec:setup}

\begin{arxiv}
We restrict our setting to binary classification,
where rule lists are Boolean functions;
this framework is straightforward to generalize to multi-class classification.
\end{arxiv}
\begin{kdd}
We restrict our setting to binary classification.
\end{kdd}
%
Let~${\{(x_n, y_n)\}_{n=1}^N}$ denote training data,
where ${x_n \in \{0, 1\}^J}$ are binary features and ${y_n \in \{0, 1\}}$ are labels.
%
Let~${\x = \{x_n\}_{n=1}^N}$ and~${\y = \{y_n\}_{n=1}^N}$,
and let~${x_{n,j}}$ denote the $j$-th feature of~$x_n$.

A rule list ${\RL = (r_1, r_2, \dots, r_K, r_0)}$ of length~${K \ge 0}$
is a ${(K+1)}$-tuple consisting of~$K$ distinct association rules,
${r_k = p_k \rightarrow q_k}$, for ${k = 1, \dots, K}$,
followed by a default rule~$r_0$.
%
\begin{arxiv}
Figure~\ref{fig:rule-list-symbols} illustrates
a rule list, ${\RL =}$ ${(r_1, r_2, r_3, r_0)}$,
which for clarity, we sometimes call a $K$-rule list.
\end{arxiv}
\begin{kdd}
Figure~\ref{fig:rule-list} illustrates a 3-rule list,
${\RL =}$ ${(r_1, r_2, r_3, r_0)}$.
\end{kdd}
%
An association rule~${r = p \rightarrow q}$ is an implication
corresponding to the conditional statement, ``if~$p$, then~$q$.''
%
In our setting, an antecedent~$p$ is a Boolean assertion that
evaluates to either true or false for each datum~$x_n$,
and a consequent~$q$ is a label prediction.
%
For example, ${(x_{n, 1} = 0) \wedge (x_{n, 3} = 1) \rightarrow}$ ${(y_n = 1)}$
is an association rule.
%
%The number of conditions in an antecedent is its cardinality;
%the antecedent in the previous example has a cardinality of two.
%
The final default rule~$r_0$ in a rule list can be thought of
as a
\begin{arxiv}
special
\end{arxiv}
association rule~${p_0 \rightarrow q_0}$
whose antecedent~$p_0$ simply asserts true.

\begin{arxiv}
\begin{figure}[t!]
\small
\begin{subfigure}{0.67\textwidth}
\begin{algorithmic}
\State \bif $(age=18-20) \band (sex=male)$ \bthen $yes$
\State \belif $(age=21-23)	 \band (priors=2-3)$ \bthen $yes$
\State \belif $(priors>3)$ \bthen $yes$
\State \belse $no$
\end{algorithmic}
\end{subfigure}
\hfill
\begin{subfigure}{0.32\textwidth}
\begin{algorithmic}
\State \bif $p_1$ \bthen $q_1$
\State \belif $p_2$ \bthen $q_2$
\State \belif $p_3$ \bthen $q_3$
\State \belse $q_0$
\end{algorithmic}
\end{subfigure}
\caption{The rule list ${\RL = (r_1, r_2, r_3, r_0)}$.
Each rule is of the form ${r_k = p_k \rightarrow q_k}$,
for all ${k = 0, \dots, 3}$.
We can also express this rule list as ${\RL = (\Prefix, \Labels, \Default, K)}$,
where ${\Prefix = (p_1, p_2, p_3)}$, ${\Labels = (1, 1, 1, 1)}$,
${\Default = 0}$, and ${K=3}$.
This is the same 3-rule list as in Figure~\ref{fig:rule-list},
that predicts two-year recidivism for the ProPublica data set.
}
\label{fig:rule-list-symbols}
\end{figure}
\end{arxiv}

Let ${\RL = (r_1, r_2, \dots, r_K, r_0)}$ be a
\begin{arxiv}
$K$-rule list,
\end{arxiv}
\begin{kdd}
rule list,
\end{kdd}
where ${r_k = p_k \rightarrow q_k}$ for each ${k = 0, \dots, K}$.
%
We introduce a useful alternate rule list representation:
${\RL = (\Prefix, \Labels, \Default, K)}$,
where we define ${\Prefix = (p_1, \dots, p_K)}$ to be $\RL$'s prefix,
${\Labels = (q_1, \dots, q_K) \in \{0, 1\}^K}$~gives
the label predictions associated with~$\Prefix$,
and ${\Default \in \{0, 1\}}$ is the default label prediction.
%
For
\begin{arxiv}
example, for
\end{arxiv}
the rule list in Figure~\ref{fig:rule-list},
we would write ${\RL = (\Prefix, \Labels, \Default, K)}$,
where ${\Prefix = (p_1, p_2, p_3)}$, ${\Labels = (1, 1, 1)}$,
${\Default = 0}$, and ${K=3}$.
%
Note that ${((), (), q_0, 0)}$ is a well-defined rule list with an empty prefix;
it is completely defined by a single default rule.

Let ${\Prefix = (p_1, \dots, p_k, \dots, p_K)}$ be an antecedent list,
then for any ${k \le K}$, we define ${\Prefix^k =}$ ${(p_1, \dots, p_k)}$
to be the $k$-prefix of~$\Prefix$.
%
For any such $k$-prefix~$\Prefix^k$,
we say that~$\Prefix$ starts with~$\Prefix^k$.
%
For any given space of rule lists,
we define~$\StartsWith(\Prefix)$ to be the set of
all rule lists whose prefixes start with~$\Prefix$:
\begin{align}
\StartsWith(\Prefix) =
\{(\Prefix', \Labels', \Default', K') : \Prefix' \textnormal{ starts with } \Prefix \}.
\label{eq:starts-with}
\end{align}
%We also say that an antecedent list~$\Prefix$ contains another
%antecedent list~$\Prefix'$ if the antecedents in~$\Prefix'$ correspond to
% a contiguous subsequence of antecedents anywhere in~$\Prefix$.
%
If ${\Prefix = (p_1, \dots, p_K)}$ and ${\Prefix' = (p_1, \dots, p_K, p_{K+1})}$
are two prefixes such that~$\Prefix'$ starts with~$\Prefix$ and extends it by
a single antecedent, we say that~$\Prefix$ is the parent of~$\Prefix'$
and that~$\Prefix'$ is a child of~$\Prefix$.

A rule list~$\RL$ classifies datum~$x_n$ by providing the label prediction~$q_k$
of the first rule~$r_k$ whose antecedent~$p_k$ is true for~$x_n$.
%
We say that an antecedent~$p_k$ of antecedent list~$\Prefix$ captures~$x_n$
in the context of~$\Prefix$ if~$p_k$ is the first antecedent in~$\Prefix$ that
evaluates to true for~$x_n$.
%
\begin{arxiv}
We also say that a
\end{arxiv}
\begin{kdd}
A
\end{kdd}
prefix captures those data captured by its antecedents;
for a rule list~${\RL = (\Prefix, \Labels, \Default, K)}$,
data not captured by the prefix~$\Prefix$
are classified according to the default label prediction~$\Default$.

Let~$\beta$ be a set of antecedents.
%
We define~${\Cap(x_n, \beta) = 1}$ if an antecedent in~$\beta$
captures datum~$x_n$, and~0 otherwise.
%
For example, let~$\Prefix$ and~$\Prefix'$ be prefixes such that~$\Prefix'$ starts
with~$\Prefix$, then~$\Prefix'$ captures all the data that~$\Prefix$ captures:
\begin{arxiv}
\begin{align*}
\{x_n: \Cap(x_n, \Prefix)\} \subseteq \{x_n: \Cap(x_n, \Prefix')\}.
%\label{eq:cap-subset}
\end{align*}
\end{arxiv}
\begin{kdd}
${\{x_n: \Cap(x_n, \Prefix)\} \subseteq \{x_n: \Cap(x_n, \Prefix')\}}$.
\end{kdd}
%We also define ${\Cap(\x, \beta) = \{\Cap(x_n, \beta)\}_{n=1}^N} \in \{0, 1\}^N$
%to be~$\beta$'s captures vector.

Now let~$\Prefix$ be an ordered list of antecedents,
and let~$\beta$ be a subset of antecedents in~$\Prefix$.
%
Let us define~${\Cap(x_n, \beta \given \Prefix) = 1}$ if~$\beta$
captures datum~$x_n$ in the context of~$\Prefix$,
\ie if the first antecedent in~$\Prefix$ that evaluates to true for~$x_n$
is an antecedent in~$\beta$, and~0 otherwise.
%
Thus, ${\Cap(x_n, \beta \given \Prefix) = 1}$ only if ${\Cap(x_n, \beta) = 1}$;
${\Cap(x_n, \beta \given \Prefix) = 0}$ either if ${\Cap(x_n, \beta) = 0}$,
or if ${\Cap(x_n, \beta) = 1}$ but there is an antecedent~$\alpha$ in~$\Prefix$,
preceding all antecedents in~$\beta$, such that ${\Cap(x_n, \alpha) = 1}$.
%
For example, if ${\Prefix = (p_1, \dots, p_k, \dots, p_K)}$ is a prefix, then
\begin{align*}
\Cap(x_n, p_k \given \Prefix) =
  \left(\bigwedge_{k'=1}^{k - 1} \neg\, \Cap(x_n, p_{k'}) \right)
  \wedge \Cap(x_n, p_k)
\end{align*}
indicates whether antecedent~$p_k$ captures datum~$x_n$ in the context of~$\Prefix$.
%
Now, define ${\Supp(\beta, \x)}$ to be the normalized support of~$\beta$,
\begin{align}
\Supp(\beta, \x) = \frac{1}{N} \sum_{n=1}^N \Cap(x_n, \beta),
\label{eq:support}
\end{align}
and similarly define~${\Supp(\beta, \x \given \Prefix)}$
to be the normalized support of~$\beta$ in the context of~$\Prefix$,
\begin{align}
\Supp(\beta, \x \given \Prefix) = \frac{1}{N} \sum_{n=1}^N \Cap(x_n, \beta \given \Prefix),
\label{eq:support-context}
\end{align}

Next, we address how empirical data constrains rule lists.
%
Given training data~${(\x, \y)}$,
an antecedent list ${\Prefix = (p_1, \dots, p_K)}$
implies a rule list ${\RL = (\Prefix, \Labels, \Default, K)}$
with prefix~$\Prefix$, where the label predictions
${\Labels = (q_1, \dots, q_K)}$ and~$\Default$ are empirically set
to minimize the number of misclassification errors made by
the rule list on the training data.
%
Thus for~${1 \le k \le K}$, label prediction~$q_k$ corresponds to the
majority label of data captured by antecedent~$p_k$ in the context of~$\Prefix$,
and the default~$\Default$ corresponds to the majority label of data
not captured by~$\Prefix$.
%
In the remainder of our presentation, whenever we refer to a rule list with a
particular prefix, we implicitly assume these empirically determined label predictions.

Our method is technically an associative classification method since it
leverages pre-mined rules.

\subsection{Objective Function}
\label{sec:objective}

\begin{arxiv}
We define
\end{arxiv}
\begin{kdd}
Define
\end{kdd}
a simple objective function for a rule list ${\RL = (\Prefix, \Labels, \Default, K)}$:
\begin{align}
\Obj(\RL, \x, \y) = \Loss(\RL, \x, \y) + \Reg K.
\label{eq:objective}
\end{align}
This objective function is a regularized empirical risk;
it consists of a loss~$\Loss(\RL, \x, \y)$, measuring misclassification error,
and a regularization term that penalizes longer rule lists.
%
$\Loss(\RL, \x, \y)$~is the fraction of training data whose labels are
incorrectly predicted by~$\RL$.
%
In our setting, the regularization parameter~${\Reg \ge 0}$ is a small constant;
\eg ${\Reg = 0.01}$ can be thought of as adding a penalty equivalent to misclassifying~$1\%$
of data when increasing a rule list's length
\begin{arxiv}
by one association rule.
\end{arxiv}
\begin{kdd}
by~one.
\end{kdd}
%
%As noted in~\S\ref{sec:setup}, a prefix~$\Prefix$ and training data together
%fully specify a rule list~${\RL = (\Prefix, \Labels, \Default, K)}$,
%thus let us define~${\Obj(\Prefix, \x, \y) \equiv \Obj(\RL, \x, \y)}$.

\subsection{Optimization Framework}
\label{sec:optimization}

Our objective has structure amenable to global optimization via a branch-and-bound framework.
%
In particular, we make a series of important observations, each of which translates into
a useful bound, and that together interact to eliminate large parts of the search~space.
%
We discuss these in depth in what follows:
%
\begin{itemize}
\item Lower bounds on a prefix also hold for every extension of that prefix.
(\S\ref{sec:hierarchical}, Theorem~\ref{thm:bound})

\item If a rule list is not accurate enough with respect to its length,
we can prune all extensions of it.
(\S\ref{sec:hierarchical}, Lemma~\ref{lemma:lookahead})

\item We can calculate \emph{a priori} an upper bound on the maximum length
of an optimal rule list.
(\S\ref{sec:ub-prefix-length}, Theorem~\ref{thm:ub-prefix-specific})

\item Each rule in an optimal rule list must have support that is sufficiently large.
%(Otherwise it would not be in an optimal rule list.)
%
This allows us to construct rule lists from frequent itemsets,
while preserving the guarantee that we can find a globally optimal
rule list from pre-mined rules.
(\S\ref{sec:lb-support}, Theorem~\ref{thm:min-capture})

\item Each rule in an optimal rule list must predict accurately.
%
In particular, the number of observations predicted correctly
by each rule in an optimal rule list must be above a threshold.
(\S\ref{sec:lb-support}, Theorem~\ref{thm:min-capture-correct})

\item We need only consider the optimal permutation of antecedents in a prefix;
we can omit all other permutations.
(\S\ref{sec:equivalent}, Theorem~\ref{thm:equivalent} and Corollary~\ref{thm:permutation})

\item  If multiple observations have identical features and opposite labels,
we know that any model will make mistakes.
%
In particular, the number of mistakes on these observations will be at least
the number of observations with the minority label.
(\S\ref{sec:identical}, Theorem~\ref{thm:identical})
\end{itemize}
\begin{kdd}
We present additional theorems and all proofs in~\citep{AngelinoLaAlSeRu17}.
\end{kdd}

\subsection{Hierarchical Objective Lower Bound}
\label{sec:hierarchical}

We can decompose the misclassification error
\begin{arxiv}
in~\eqref{eq:objective}
\end{arxiv}
into two contributions corresponding to the prefix and the default rule:
\begin{align*}
\Loss(\RL, \x, \y) %= \Loss(\Prefix, r_q, \Default, \x, \y)
\equiv \Loss_p(\Prefix, \Labels, \x, \y) + \Loss_0(\Prefix, \Default, \x, \y),
\end{align*}
where ${\Prefix = (p_1, \dots, p_K)}$ and ${\Labels = (q_1, \dots, q_K)}$;
\begin{align*}
\Loss_p(\Prefix, \Labels, \x, \y) =
\frac{1}{N} \sum_{n=1}^N \sum_{k=1}^K \Cap(x_n, p_k \given \Prefix) \wedge \one [ q_k \neq y_n ]
%\label{eq:loss}
\end{align*}
is the fraction of data captured and misclassified by the prefix, and
\begin{align*}
\Loss_0(\Prefix, \Default, \x, \y) =
\frac{1}{N} \sum_{n=1}^N \neg\, \Cap(x_n, \Prefix) \wedge \one [ \Default \neq y_n ]
\end{align*}
is the fraction of data not captured by the prefix and misclassified by the default rule.
%
Eliminating the latter error term gives a lower bound~$b(\Prefix, \x, \y)$ on the objective,
\begin{align}
b(\Prefix, \x, \y) \equiv \Loss_p(\Prefix, \Labels, \x, \y) + \Reg K \le \Obj(\RL, \x, \y),
\label{eq:lower-bound}
\end{align}
where we have suppressed the lower bound's dependence on label predictions~$\Labels$
because they are fully determined, given~${(\Prefix, \x, \y)}$.
%
Furthermore,
\begin{arxiv}
as we state next in Theorem~\ref{thm:bound},
\end{arxiv}
$b(\Prefix, \x, \y)$ gives a lower bound on the objective of
\emph{any} rule list whose prefix starts with~$\Prefix$.

\begin{theorem}[Hierarchical objective lower bound]
\begin{arxiv}
Define~${b(\Prefix, \x, \y)}$
\end{arxiv}
\begin{kdd}
Define~${b(\Prefix, \x, \y) = \Loss_p(\Prefix, \Labels, \x, \y) + \Reg K}$,
\end{kdd}
as in~\eqref{eq:lower-bound}.
%
Also, define $\StartsWith(\Prefix)$ to be the set of all rule lists
whose prefixes starts with~$\Prefix$, as in~\eqref{eq:starts-with}.
%
Let ${\RL = }$ ${(\Prefix, \Labels, \Default, K)}$ be a rule list
with prefix~$\Prefix$, and let
${\RL' = (\Prefix', \Labels', \Default', K')}$ $\in \StartsWith(\Prefix)$
be any rule list such that its prefix~$\Prefix'$ starts with~$\Prefix$
and ${K' \ge K}$, then ${b(\Prefix, \x, \y) \le}$ ${\Obj(\RL', \x, \y)}$.
\label{thm:bound}
\end{theorem}

\begin{arxiv}
\begin{proof}
Let ${\Prefix = (p_1, \dots, p_K)}$ and ${\Labels = (q_1, \dots, q_K)}$;
let ${\Prefix' = (p_1, \dots, p_K, p_{K+1}, \dots, p_{K'})}$
and ${\Labels' = (q_1, \dots, q_K, q_{K+1}, \dots, q_{K'})}$.
%
Notice that~$\Prefix'$ yields the same mistakes as~$\Prefix$,
and possibly additional mistakes:
\begin{align}
&\Loss_p(\Prefix', \Labels', \x, \y)
= \frac{1}{N} \sum_{n=1}^N  \sum_{k=1}^{K'} \Cap(x_n, p_k \given \Prefix') \wedge \one [ q_k \neq y_n ] \nn \\
&= \frac{1}{N} \sum_{n=1}^N \left( \sum_{k=1}^K \Cap(x_n, p_k \given \Prefix) \wedge \one [ q_k \neq y_n ]
+ \sum_{k=K+1}^{K'} \Cap(x_n, p_k \given \Prefix') \wedge \one [ q_k \neq y_n ] \right) \nn \\
&=\Loss_p(\Prefix, \Labels, \x, \y)
+ \frac{1}{N} \sum_{n=1}^N \sum_{k=K+1}^{K'} \Cap(x_n, p_k \given \Prefix') \wedge \one [ q_k \neq y_n ]
\ge \Loss_p(\Prefix, \Labels, \x, \y),
\label{eq:prefix-loss}
\end{align}
where in the second equality we have used the fact that
${\Cap(x_n, p_k \given \Prefix') = \Cap(x_n, p_k \given \Prefix)}$
for~${1 \le k \le K}$.
%
It follows that
\begin{align}
b(\Prefix, \x, \y) &= \Loss_p(\Prefix, \Labels, \x, \y) + \Reg K \nn \\
&\le  \Loss_p(\Prefix', \Labels', \x, \y) + \Reg K' = b(\Prefix', \x, \y)
\le \Obj(\RL', \x, \y).
\label{eq:prefix-lb}
\end{align}
\end{proof}
\end{arxiv}

To generalize, consider a sequence of prefixes such that each prefix
starts with all previous prefixes in the sequence.
%
It follows that the corresponding sequence of objective lower bounds
increases monotonically.
%
This is precisely the structure required and exploited by branch-and-bound,
illustrated in Algorithm~\ref{alg:branch-and-bound}.

\begin{algorithm}[t!]
\caption{Branch-and-bound for learning rule lists.}
\label{alg:branch-and-bound}
\begin{algorithmic}
\normalsize
\State \textbf{Input:} Objective function $\Obj(\RL, \x, \y)$,
objective lower bound ${b(\Prefix, \x, \y)}$,
set of antecedents ${\RuleSet = \{s_m\}_{m=1}^M}$,
training data $(\x, \y) = {\{(x_n, y_n)\}_{n=1}^N}$,
initial best known rule list~$\InitialRL$ with objective
${\InitialObj = \Obj(\InitialRL, \x, \y)}$;
$\InitialRL$ could be obtained as output from another (approximate) algorithm,
otherwise, $(\InitialRL, \InitialObj) = (\text{null}, 1)$ provide reasonable default values
\State \textbf{Output:} Provably optimal rule list~$\OptimalRL$ with minimum objective~$\OptimalObj$ \\

\State $(\CurrentRL, \CurrentObj) \gets (\InitialRL, \InitialObj)$ \Comment{Initialize best rule list and objective}
\State $Q \gets $ queue$(\,[\,(\,)\,]\,)$ \Comment{Initialize queue with empty prefix}
\While {$Q$ not empty} \Comment{Stop when queue is empty}
	\State $\Prefix \gets Q$.pop(\,) \Comment{Remove prefix~$\Prefix$ from the queue}
	\begin{arxiv}
	\State $\RL \gets (\Prefix, \Labels, \Default, K)$ \Comment{Set label predictions~$\Labels$ and~$\Default$ to minimize training error}
	\end{arxiv}
	\If {$b(\Prefix, \x, \y) < \CurrentObj$} \Comment{\textbf{Bound}: Apply Theorem~\ref{thm:bound}}
        \State $\Obj \gets \Obj(\RL, \x, \y)$ \Comment{Compute objective of~$\Prefix$'s rule list~$\RL$}
        \If {$\Obj < \CurrentObj$} \Comment{Update best rule list and objective}
            \State $(\CurrentRL, \CurrentObj) \gets (\RL, \Obj)$
        \EndIf
        \For {$s$ in $\RuleSet$} \Comment{\textbf{Branch}: Enqueue~$\Prefix$'s children}
            \If {$s$ not in $\Prefix$}
                \State $Q$.push$(\,(\Prefix, s)\,)$
            \EndIf
        \EndFor
    \EndIf
\EndWhile
\State $(\OptimalRL, \OptimalObj) \gets (\CurrentRL, \CurrentObj)$ \Comment{Identify provably optimal solution}
\end{algorithmic}
\end{algorithm}

Specifically, the objective lower bound in Theorem~\ref{thm:bound}
enables us to prune the state space hierarchically.
%
While executing branch-and-bound, we keep track of the current best (smallest)
objective~$\CurrentObj$, thus it is a dynamic, monotonically decreasing quantity.
%
If we encounter a prefix~$\Prefix$ with lower bound
${b(\Prefix, \x, \y) \ge \CurrentObj}$,
then by Theorem~\ref{thm:bound}, we do not need to consider \emph{any}
rule list~${\RL' \in \StartsWith(\Prefix)}$ whose prefix~$\Prefix'$ starts with~$\Prefix$.
%because ${b(\Prefix', \x, \y) \ge b(\Prefix, \x, \y)}$. \\
%
For the objective of such a rule list, the current best objective
provides a lower bound, \ie
${\Obj(\RL', \x, \y) \ge b(\Prefix', \x, \y) \ge}$ ${b(\Prefix, \x, \y) \ge \CurrentObj}$,
and thus~$\RL'$ cannot be optimal.
