\begin{kdd}
\section{Implementation}
\label{sec:implementation}
\end{kdd}

\begin{arxiv}
\section{Incremental computation}
\label{sec:incremental}
\end{arxiv}

For every prefix~$\Prefix$ evaluated during
Algorithm~\ref{alg:branch-and-bound}'s execution, we compute
the objective lower bound~${b(\Prefix, \x, \y)}$ and sometimes
the objective~${\Obj(\RL, \x, \y)}$ of the corresponding rule list~$\RL$.
%
These calculations are the dominant
\begin{arxiv}
computations with respect to execution time.
%
This motivates
\end{arxiv}
\begin{kdd}
computations, and motivate
\end{kdd}
our use of a highly optimized library,
designed by~\citet{YangRuSe16} for representing rule lists and
performing operations encountered in evaluating functions of rule lists.
%
Furthermore, we exploit the hierarchical nature of the objective
function and its lower bound to compute these quantities
incrementally throughout branch-and-bound execution.
%
\begin{arxiv}
In this section, we provide explicit expressions for
the incremental computations that are central to our approach.
%
Later, in~\S\ref{sec:implementation}, we describe a cache data structure
for supporting our incremental framework in practice.

For completeness, before presenting our incremental expressions,
let us begin by writing down the objective lower bound and objective
of the empty rule list, ${\RL = ((), (), \Default, 0)}$,
the first rule list evaluated in Algorithm~\ref{alg:branch-and-bound}.
%
Since its prefix contains zero rules, it has zero prefix
misclassification error and also has length zero.
%
Thus, the empty rule list's objective lower bound is zero, \ie ${b((), \x, \y) = 0}$.
%
Since none of the data are captured by the empty prefix, the default rule
corresponds to the majority class, and the objective corresponds to the
default rule misclassification error, \ie ${\Obj(\RL, \x, \y) = \Loss_0((), \Default, \x, \y)}$.

Now, we derive our incremental expressions for the objective function and its lower bound.
%
Let ${\RL = (\Prefix, \Labels, \Default, K)}$ and
${\RL' = (\Prefix', \Labels', \Default', K + 1)}$
be rule lists such that prefix ${\Prefix =}$ ${(p_1, \dots, p_K)}$
is the parent of ${\Prefix' = (p_1, \dots, p_K, p_{K+1})}$.
%
Let ${\Labels = (q_1, \dots, q_K)}$ and
${\Labels' = (q_1, \dots,}$ ${q_K, q_{K+1})}$ be the corresponding labels.
%
The hierarchical structure of Algorithm~\ref{alg:branch-and-bound}
enforces that if we ever evaluate~$\RL'$, then we will have already
evaluated both the objective and objective lower bound of its parent,~$\RL$.
%
We would like to reuse as much of these computations as possible
in our evaluation of~$\RL'$.
%
We can write the objective lower bound of~$\RL'$ incrementally,
with respect to the objective lower bound of~$\RL$:
\begin{align}
b(\Prefix', \x, \y)
  &= \Loss_p(\Prefix', \Labels', \x, \y) + \Reg (K + 1) \nn \\
&= \frac{1}{N} \sum_{n=1}^N \sum_{k=1}^{K+1} \Cap(x_n, p_k \given \Prefix')
  \wedge \one [ q_k \neq y_n ] + \Reg (K+1) \label{eq:non-inc-lb} \\
&= \Loss_p(\Prefix, \Labels, \x, \y) + \Reg K + \Reg
  + \frac{1}{N} \sum_{n=1}^N \Cap(x_n, p_{K+1} \given \Prefix') \wedge \one [q_{K+1} \neq y_n ] \nn \\
&= b(\Prefix, \x, \y) + \Reg
  + \frac{1}{N} \sum_{n=1}^N \Cap(x_n, p_{K+1} \given \Prefix') \wedge \one [q_{K+1} \neq y_n ] \nn \\
&= b(\Prefix, \x, \y) + \Reg  + \frac{1}{N} \sum_{n=1}^N \neg\, \Cap(x_n, \Prefix) \wedge
  \Cap(x_n, p_{K+1}) \wedge \one [q_{K+1} \neq y_n].
\label{eq:inc-lb}
\end{align}
%Notice that the term~${\neg\, \Cap(x_n, \Prefix)}$ in~\eqref{eq:inc-lb}
%depends only on~$\Prefix$, \ie it does not depend on~$\Prefix'$; furthermore,
%it is computed in the evaluation of~$\RL$'s default rule misclassification error,
%\begin{align}
%\Loss_0(\Prefix, \Default, \x, \y) = \frac{1}{N}\sum_{n=1}^N \neg\, \Cap(x_n, \Prefix) \wedge \one [q_0 \neq y_n].
%\end{align}
Thus, if we store $b(\Prefix, \x, \y)$, % and ${\neg\, \Cap(\x, \Prefix)}$,
then we can reuse this quantity when computing $b(\Prefix', \x, \y)$.
%
Transforming~\eqref{eq:non-inc-lb} into~\eqref{eq:inc-lb} yields a
significantly simpler expression that is a function of the stored
quantity~$b(\Prefix, \x, \y)$. %, as well as ${\neg\, \Cap(\x, \Prefix)}$
%and the last rule of~$\RL'$, ${p_{K+1} \rightarrow q_{K+1}}$.
%
For the objective of~$\RL'$, first let us write a na\"ive expression:
\begin{align}
&\Obj(\RL', \x, \y) = \Loss(\RL', \x, \y) + \Reg (K + 1)
= \Loss_p(\Prefix', \Labels', \x, \y) + \Loss_0(\Prefix', \Default', \x, \y) + \Reg(K + 1) \nn \\
&= \frac{1}{N} \sum_{n=1}^N \sum_{k=1}^{K+1} \Cap(x_n, p_k \given \Prefix')
  \wedge \one [ q_k \neq y_n ] + \frac{1}{N}\sum_{n=1}^N \neg\, \Cap(x_n, \Prefix') \wedge
  \one [q'_0 \neq y_n] + \Reg (K+1). \label{eq:non-inc-obj}
\end{align}
Instead, we can compute the objective of~$\RL'$ incrementally
with respect to its objective lower bound:
\begin{align}
\Obj(\RL', \x, \y) &=  \Loss_p(\Prefix', \Labels', \x, \y) +
  \Loss_0(\Prefix', \Default', \x, \y) + \Reg (K + 1) \nn \\
&= b(\Prefix', \x, \y) + \Loss_0(\Prefix', \Default', \x, y) \nn \\
&= b(\Prefix', \x, \y) + \frac{1}{N}\sum_{n=1}^N \neg\, \Cap(x_n, \Prefix') \wedge
  \one [q'_0 \neq y_n] \nn \\
&= b(\Prefix', \x, \y) + \frac{1}{N}\sum_{n=1}^N \neg\, \Cap(x_n, \Prefix) \wedge
  (\neg\, \Cap(x_n, p_{K+1})) \wedge \one [q'_0 \neq y_n].
\label{eq:inc-obj}
\end{align}
The expression in~\eqref{eq:inc-obj} is simpler to compute than that
in~\eqref{eq:non-inc-obj}, because the former reuses $b(\Prefix', \x, \y)$,
which we already computed in~\eqref{eq:inc-lb}.
%as well as ${\neg\, \Cap(\x, \Prefix)}$,
%and the last antecedent~$p_{K+1}$ and default rule~$\Default'$ of~$\RL'$.
Note that instead of computing~$\Obj(\RL', \x, \y)$ incrementally from $b(\Prefix', \x, \y)$
as in~\eqref{eq:inc-obj}, we could have computed it incrementally from $\Obj(\RL, \x, \y)$.
However, doing so would in practice require that we store~$\Obj(\RL, \x, \y)$
in addition to~$b(\Prefix, \x, \y)$, which we already must store to support~\eqref{eq:inc-lb}.
We prefer the incremental approach suggested by~\eqref{eq:inc-obj}
since it avoids this additional storage overhead.

\begin{algorithm}[t!]
  \caption{Incremental branch-and-bound for learning rule lists, for simplicity, from a cold start.
  We explicitly show the incremental objective lower bound and objective functions in Algorithms~\ref{alg:incremental-lb} and~\ref{alg:incremental-obj}, respectively.}
\label{alg:incremental}
\begin{algorithmic}
\normalsize
\State \textbf{Input:} Objective function~$\Obj(\RL, \x, \y)$,
objective lower bound~${b(\Prefix, \x, \y)}$,
set of antecedents ${\RuleSet = \{s_m\}_{m=1}^M}$,
training data~$(\x, \y) = {\{(x_n, y_n)\}_{n=1}^N}$,
regularization parameter~$\Reg$
\State \textbf{Output:} Provably optimal rule list~$\OptimalRL$ with minimum objective~$\OptimalObj$ \\

\State $\CurrentRL \gets ((), (), \Default, 0)$ \Comment{Initialize current best rule list with empty rule list}
\State $\CurrentObj \gets \Obj(\CurrentRL, \x, \y)$ \Comment{Initialize current best objective}
\State $Q \gets $ queue$(\,[\,(\,)\,]\,)$ \Comment{Initialize queue with empty prefix}
\State $C \gets $ cache$(\,[\,(\,(\,)\,, 0\,)\,]\,)$ \Comment{Initialize cache with empty prefix and its objective lower bound}
\While {$Q$ not empty} \Comment{Optimization complete when the queue is empty}
	\State $\Prefix \gets Q$.pop(\,) \Comment{Remove a length-$K$ prefix~$\Prefix$ from the queue}
        \State $b(\Prefix, \x, \y) \gets C$.find$(\Prefix)$ \Comment{Look up $\Prefix$'s lower bound in the cache}
        \State $\mathbf{u} \gets \neg\,\Cap(\x, \Prefix)$ \Comment{Bit vector indicating data not captured by $\Prefix$}
        \For {$s$ in $\RuleSet$} \Comment{Evaluate all of $\Prefix$'s children}
            \If {$s$ not in $\Prefix$}
                \State $\PrefixB \gets (\Prefix, s)$ \Comment{\textbf{Branch}: Generate child $\PrefixB$}
                \State $\mathbf{v} \gets \mathbf{u} \wedge \Cap(\x, s)$ \Comment{Bit vector indicating data captured by $s$ in $\PrefixB$}
                \State $b(\PrefixB, \x, \y) \gets b(\Prefix, \x, \y) + \Reg~ + $ \Call{IncrementalLowerBound}{$\mathbf{v}, \y, N$} %\Comment{Eq.~\eqref{eq:inc-lb}}
                \If {$b(\PrefixB, \x, \y) < \CurrentObj$} \Comment{\textbf{Bound}: Apply bound from Theorem~\ref{thm:bound}}
                    \State $\Obj(\RLB, \x, \y) \gets b(\PrefixB, \x, \y)~ + $ \Call{IncrementalObjective}{$\mathbf{u}, \mathbf{v}, \y, N$} %\Comment{Eq.~\eqref{eq:inc-obj}}
                    \If {$\Obj(\RLB, \x, \y) < \CurrentObj$}
                        \State $\RLB \gets (\PrefixB, \LabelsB, \DefaultB, K+1)$ \Comment{$\LabelsB, \DefaultB$ are set in the incremental functions}
                        \State $(\CurrentRL, \CurrentObj) \gets (\RLB, \Obj(\RLB, \x, \y))$ \Comment{Update current best rule list and objective}
                    \EndIf
                    \State $Q$.push$(\PrefixB)$ \Comment{Add $\PrefixB$ to the queue}
                    \State $C$.insert$(\PrefixB, b(\PrefixB, \x, \y))$ \Comment{Add $\PrefixB$ and its lower bound to the cache}
                \EndIf
            \EndIf
        \EndFor
\EndWhile
\State $(\OptimalRL, \OptimalObj) \gets (\CurrentRL, \CurrentObj)$ \Comment{Identify provably optimal rule list and objective}
\end{algorithmic}
\end{algorithm}

\begin{algorithm}[t!]
  \caption{Incremental objective lower bound~\eqref{eq:inc-lb} used in Algorithm~\ref{alg:incremental}.}
\label{alg:incremental-lb}
\begin{algorithmic}
\normalsize
\State \textbf{Input:}
Bit vector~${\mathbf{v} \in \{0, 1\}^N}$ indicating data captured by $s$, the last antecedent in~$\PrefixB$,
bit vector of class labels~${\y \in \{0, 1\}^N}$,
number of observations~$N$
\State \textbf{Output:} Component of~$\RLB$'s misclassification error due to data captured by~$s$ \\

\Function{IncrementalLowerBound}{$\mathbf{v}, \y, N$}
    \State $n_v = \Count(\mathbf{v})$ \Comment{Number of data captured by $s$, the last antecedent in $\PrefixB$}
    \State $\mathbf{w} \gets \mathbf{v} \wedge \y$ \Comment{Bit vector indicating data captured by $s$ with label $1$}
    \State $n_w = \Count(\mathbf{w})$ \Comment{Number of data captured by $s$ with label $1$}
    \If {$n_w / n_v > 0.5$}
        \State \Return $(n_v - n_w) / N$ \Comment{Misclassification error of the rule $s \rightarrow 1$}
    \Else
        \State \Return $n_w / N$ \Comment{Misclassification error of the rule $s \rightarrow 0$}
    \EndIf
    \EndFunction
\end{algorithmic}
\end{algorithm}

\begin{algorithm}[t!]
  \caption{Incremental objective function~\eqref{eq:inc-obj} used in Algorithm~\ref{alg:incremental}.}
\label{alg:incremental-obj}
\begin{algorithmic}
\normalsize
\State \textbf{Input:}
Bit vector~${\mathbf{u} \in \{0, 1\}^N}$ indicating data not captured by~$\PrefixB$'s parent prefix,
bit vector~${\mathbf{v} \in \{0, 1\}^N}$ indicating data not captured by $s$, the last antecedent in~$\PrefixB$,
bit vector of class labels~${\y \in \{0, 1\}^N}$,
number of observations~$N$
\State \textbf{Output:} Component of~$\RLB$'s misclassification error due to its default rule \\

 \Function{IncrementalObjective}{$\mathbf{u}, \mathbf{v}, \y, N$}
    \State $\mathbf{f} \gets \mathbf{u} \wedge \neg\,\mathbf{v} $ \Comment{Bit vector indicating data not captured by $\PrefixB$}
    \State $n_f = \Count(\mathbf{f})$ \Comment{Number of data not captured by $\PrefixB$}
    \State $\mathbf{g} \gets \mathbf{f} \wedge \y$ \Comment{Bit vector indicating data not captured by $\PrefixB$ with label $1$}
    \State $n_g = \Count(\mathbf{g})$ \Comment{Number of data not captued by $\PrefixB$ with label $1$}
    \If {$n_g / n_f > 0.5$}
        \State \Return $(n_f - n_g) / N$ \Comment{Misclassification error of the default label prediction $1$}
    \Else
        \State \Return $n_g / N$ \Comment{Misclassification error of the default label prediction $0$}
    \EndIf
\EndFunction
\end{algorithmic}
\end{algorithm}

We present an incremental branch-and-bound procedure in
Algorithm~\ref{alg:incremental}, and show the incremental computations
of the objective lower bound~\eqref{eq:inc-lb} and objective~\eqref{eq:inc-obj}
as two separate functions in Algorithms~\ref{alg:incremental-lb}
and~\ref{alg:incremental-obj}, respectively.
%
In Algorithm~\ref{alg:incremental}, we use a cache to store
prefixes and their objective lower bounds.
%
Algorithm~\ref{alg:incremental} additionally reorganizes the structure
of Algorithm~\ref{alg:branch-and-bound} to group together the computations
associated with all children of a particular prefix.
%
This has two advantages.
%
The first is to consolidate cache queries: all children of the same
parent prefix compute their objective lower bounds with respect to
the parent's stored value, and we only require one cache `find' operation
for the entire group of children, instead of a separate query for each child.
%
The second is to shrink the queue's size:
instead of adding all of a prefix's children as separate queue elements,
we represent the entire group of children in the queue by a single element.
%
Since the number of children associated with each prefix
is close to the total number of possible antecedents,
both of these effects can yield significant savings.
%
For example, if we are trying to optimize over rule lists formed
from a set of 1000 antecedents, then the maximum queue size in
Algorithm~\ref{alg:incremental} will be smaller than that in
Algorithm~\ref{alg:branch-and-bound} by a factor of nearly 1000.

\end{arxiv}


\begin{arxiv}
\section{Implementation}
\label{sec:implementation}
\end{arxiv}

We implement our algorithm using a collection of optimized data structures:
a trie (prefix tree), a symmetry-aware map, and a queue.
The trie acts like a cache, keeping track of rule lists we have already evaluated.
Each node in the trie contains metadata associated with that corresponding rule list;
the metadata consists of bookkeeping information such as what child rule lists are feasible and
the lower bound and accuracy for that rule list.
We also track the best observed minimum objective and its associated rule list.

The symmetry-aware map supports symmetry-aware pruning.
%
We implement this using the C++ STL unordered\_map,
% We have two different versions of the map.
to map all permutations of a set of antecedents to a key, whose value
contains the best ordering of those antecedents (\ie the prefix with the smallest lower bound).
%
Every antecedent is associated with an index, and we call the numerically
sorted order of a set of antecedents its canonical order.
%
Thus by querying a set of antecedents by its canonical order, all
permutations map to the same key.
% Keys in one version of the map represent the set of rules (in canonical order) comprising a
% rule list prefix.
% Keys in the other version represent the set of captured data points.
% The set of captured entries is identical for a given set of rules, independent of ordering, so
% different permutations still map to the same key.
%
% Note that encodings of rule lists in canonical order tend to be
% significantly smaller than encodings of captured data points,
% especially for large datasets.
%
\begin{kdd}
This map dominates memory usage for problems that explore longer prefixes.
\end{kdd}
\begin{arxiv}
The symmetry-aware map dominates memory usage for problems that explore longer prefixes.
\end{arxiv}
%
Before inserting permutation $P_i$ into the symmetry-aware map, we check
if there exists a permutation $P_j$ of $P_i$ already in the map.
If there is no permutation exists, then we insert $P_i$ in the map.
Otherwise, if a permutation $P_j$ exists and the lower bound of $P_i$ is better than 
that of $P_j$, we update the map and remove $P_j$ and its subtree from the trie.
Else, if $P_j$ exists and has a better lower bound than $P_i$, we do nothing 
(\ie we do not insert $P_i$ into the symmetry-aware map or the trie).

We use a queue to store all of the leaves of the trie that still need to be explored.
%
\begin{arxiv}
We order entries in the queue to implement several different policies.
%
A first-in-first-out (FIFO) queue implements breadth-first search (BFS),
and a priority queue implements best-first search.
%
Example priority queue policies include ordering
by the lower bound, the objective, or a custom metric
that maps prefixes to real values.
\end{arxiv}
\begin{kdd}
We order entries in the queue to implement several different policies,
including breadth-first search (BFS) and best-first search.
%
For best-first we use a priority queue, ordered by the lower bound, the objective,
or a custom priority metric.
\end{kdd}
%
We also support a stochastic exploration process that bypasses
the need for a queue by instead following random paths from the root to leaves.
%
We find that ordering by the lower bound and other priority metrics
often leads to a shorter runtime than using BFS.


Mapping our algorithm to our data structures produces the following execution strategy.
%
While the trie contains unexplored leaves, a scheduling policy selects the next prefix to extend.
%
Then, for every antecedent that is not already in this prefix, we calculate the lower bound,
objective, and other metrics for the rule list formed by appending the antecedent to the prefix.
%
If the lower bound of the new rule list is less than the current minimum objective, we insert that
rule list into the symmetry-aware map, trie, and queue, and, if relevant, update the
current minimum objective.
%
If the lower bound is greater than the minimum objective,
then no extension of this rule list could possibly be optimal,
thus we do not insert the new rule list into the tree or queue.
%
We also leverage our other bounds from~\S\ref{sec:framework}
to aggressively prune the search space.

During execution, we garbage collect the trie.
%
Each time we update the minimum objective,
we traverse the trie in a depth-first manner, deleting all subtrees
of any node with lower bound larger than the current minimum objective.
%
At other times, when we encounter a node with no children, we prune upwards--deleting that
node and recursively traversing the tree towards the root, deleting any childless nodes.
%
This garbage collection allows us to constrain the trie's memory consumption, though in our
experiments we observe the minimum objective to decrease only a
\begin{kdd}
few times. \\

Our code is at \textbf{\url{https://github.com/nlarusstone/corels}}.
\end{kdd}
\begin{arxiv}
small number of times. \\

Our implementation of CORELS is at \url{https://github.com/nlarusstone/corels}.
\end{arxiv}

\begin{kdd}
\vspace{-1mm}
\end{kdd}

\begin{comment}
\section{Curiosity}

We introduce a custom priority metric that we call \emph{curiosity}:
\begin{align}
\Curiosity(\Prefix, \x, \y) = \frac{1}{\NCap} \left(\sum_{n=1}^N \sum_{k=1}^K
  \Cap(x_n, p_k \given \Prefix) \wedge \one [ q_k \neq y_n ] \right)
  + \frac{\Reg K N}{\NCap} \,,
\end{align}
where~$\NCap$ is the number of datapoints captured by~$\Prefix$, \ie
${\NCap \equiv \sum_{n=1}^N \Cap(x_n, \Prefix)}$.
%
We can think of the curiosity as the expected objective value
of a rule list~${\RL' = }$ ${(\Prefix', \Labels', \Default', K')}$
generated from~$\Prefix$, for a simple model.
%
Assume that prefix~$\Prefix'$ starts with~$\Prefix$ and captures all the data,
such that each additional antecedent in~$\Prefix'$
both captures as many `new' datapoints and makes as many mistakes as
as each antecedent in~$\Prefix$, on average.
%
The expected objective value is then the sum of the expected prefix
misclassification error and the expected regularization penalty:
\begin{align}
\E[ \Obj(\RL', \x, \y) ] &= \E[\Loss_p(\Prefix, \Labels, \x, \y)] + \E[ \Reg K' ] \nn \\
&= \E[ K' ] \left(\frac{\Loss_p(\Prefix, \Labels, \x, \y)}{K}\right) + \Reg \E[ K' ] \nn \\
&=  \left(\frac{N}{\NCap / K}\right)
  \left(\frac{\Loss_p(\Prefix, \Labels, \x, \y)}{K}\right)
  + \Reg \left(\frac{N}{\NCap / K}\right) \nn \\
&= \left(\frac{N}{\NCap}\right) \left(\frac{1}{N} \sum_{n=1}^N \sum_{k=1}^K
  \Cap(x_n, p_k \given \Prefix) \wedge \one [ q_k \neq y_n ] \right)
  + \frac{\Reg K N}{\NCap} = \Curiosity(\Prefix, \x, \y).
\end{align}
\end{comment}
