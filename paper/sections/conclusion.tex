\section{Conclusion}

CORELS is an efficient and accurate algorithm for constructing provably optimal rule lists.
%
Optimality is particularly important in domains where model interpretability
has social consequences, \eg recidivism prediction.
%
While achieving optimality on such discrete optimization problems is
computationally hard in general, we aggressively prune our problem's search space
via a suite of bounds.
%
This makes realistically sized problems tractable.
%
CORELS is amenable to parallelization, which should allow it to scale to
even larger problems.

\begin{arxiv}
Finally, we would like to clarify some limitations of CORELS.
%
As far as we can tell, CORELS is the current best algorithm for solving a
specialized optimal decision tree problem.
%
It can handle large numbers of observations, but could have difficulty proving
optimality when there are many highly correlated features; this is because it is
more difficult to exclude portions of the search space when this happens.
%
It is not designed for raw image processing or other problems where the features
themselves are not interpretable.
%
It does not automatically choose the subgroup with the highest likelihood of a
positive outcome.
%
For that, one would need an algorithm such as Falling Rule Lists~\citep{WangRu15},
which forces the estimated probabilities to decrease along the list.
%
Also CORELS does not technically produce estimates of~${P(Y=1 \given x)}$ but one
could simply compute the empirical proportion~${\hat{P}(Y=1 \given x \textrm{ obeys } a_k)}$
for each prefix~$a_k$, and use this to estimate~${P(Y=1 \given x)}$.
%
CORELS is not designed to assist with causal inference applications since
it does not estimate the effect of a treatment via the conditional difference
${P(y=1 \given \textrm{treatment} = \textrm{True}, x) - P(y=1 \given \textrm{treatment} = \textrm{False}, x)}$.
%
Alternative algorithms that estimate conditional differences with interpretable
rule lists are Causal Falling Rule Lists~\citep{WangRu16}
and Cost-Effective Interpretable Treatment Regimes (CITR)~\citep{LakkarajuRu17}.
%
CORELS's rule lists are a form of decision tree.
%
In CORELS's ``trees,'' the leaves are conjunctions.
%
It may be possible to generalize ideas from CORELS to handle decision trees of
other forms, which could be an interesting project for future work.
\end{arxiv}
