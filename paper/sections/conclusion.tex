\section{Conclusion}

CORELS is an efficient and accurate algorithm for constructing provably optimal rule lists.
%
Optimality is particularly important in domains where model interpretability
has social consequences, \eg recidivism prediction.
%
While achieving optimality on such discrete optimization problems is
computationally hard in general, we aggressively prune our problem's search space
via a suite of bounds.
%
This makes realistically sized problems tractable.
%
CORELS is amenable to parallelization, which should allow it to scale to
even larger problems.

\begin{arxiv}
Finally, we would like to clarify some limitations of CORELS.
%
As far as we can tell, CORELS is the current best algorithm for solving a
specialized optimal decision tree problem.
%
While our approach scales well to large numbers of observations,
it could have difficulty proving optimality
for problems with many possibly relevant features that are highly correlated,
when large regions of the search space might be challenging to exclude.

CORELS is not designed for raw image processing or other problems where the features
themselves are not interpretable.
%
It also does not automatically choose the subgroup with the highest likelihood of a
positive outcome; doing so would require an algorithm such as Falling Rule Lists \citep{WangRu15},
which forces the estimated probabilities to decrease along the list.
%
While CORELS does not technically produce estimates of ${\P(Y=1 \given X)}$,
one could form such an estimate by computing the empirical
proportion ${\hat{\P}(Y=1 \given X \textrm{ obeys } p_k)}$ for each antecedent~$p_k$.
%
Furthermore, it is not designed to assist with causal inference applications, since
it does not estimate the effect of a treatment via the conditional difference
${\P(Y=1 \given \textrm{treatment} = \textrm{True}, X) -}$ ${\P(Y=1 \given \textrm{treatment} = \textrm{False}, X)}$.
%
Alternative algorithms that estimate conditional differences with interpretable
rule lists include Causal Falling Rule Lists \citep{WangRu16}
and Cost-Effective Interpretable Treatment Regimes (CITR) \citep{LakkarajuRu17}.

Lastly, recall that a rule list is a form of decision tree,
where in our setting, the leaves are conjunctions.
%
It may be possible to generalize ideas from CORELS to handle decision trees of
other forms, which could be an interesting project for future work.
\end{arxiv}
