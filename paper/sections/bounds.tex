Next, we state an immediate consequence of Theorem~\ref{thm:bound}.

\begin{lemma}[Objective lower bound with one-step lookahead]
\label{lemma:lookahead}
Let~$\Prefix$ be a $K$-prefix
and let~$\CurrentObj$ be the current best objective.
%
If ${b(\Prefix, \x, \y) + \Reg \ge \CurrentObj}$,
then for any $K'$-rule list ${\RL' \in \StartsWith(\Prefix)}$
whose prefix~$\Prefix'$ starts with~$\Prefix$ and~${K' > K}$,
it follows that ${\Obj(\RL', \x, \y) \ge \CurrentObj}$.
\end{lemma}

\begin{arxiv}
\begin{proof}
By the definition of the lower bound~\eqref{eq:lower-bound},
which includes the penalty for longer prefixes,
\begin{align}
\Obj(\Prefix', \x, y) \ge b(\Prefix', \x, \y) &= \Loss_p(\Prefix', \Labels', \x, \y) + \Reg K' \nn \\
&= \Loss_p(\Prefix', \Labels', \x, \y) + \Reg K + \Reg (K' - K) \nn \\
&= b(\Prefix, \x, \y) + \Reg (K' - K)
\ge b(\Prefix, \x, \y) + \Reg \ge \CurrentObj.
%\label{eq:lookahead}
\end{align}
\end{proof}
\end{arxiv}

Therefore, even if we encounter a prefix~$\Prefix$
with lower bound ${b(\Prefix, \x, \y) \le \CurrentObj}$,
\begin{kdd}
if
\end{kdd}
\begin{arxiv}
as long as
\end{arxiv}
${b(\Prefix, \x, \y) + \Reg \ge \CurrentObj}$, then we can prune
all prefixes~$\Prefix'$ that start with and are longer than~$\Prefix$.

\subsection{Upper bounds on prefix length}
\label{sec:ub-prefix-length}

\begin{arxiv}
In this section, we derive several upper bounds on prefix length:
\begin{itemize}
\item The simplest upper bound on prefix length is given by the total
number of available antecedents. (Proposition~\ref{prop:trivial-length})
%
\item The current best objective~$\CurrentObj$ implies an upper bound
on prefix length. (Theorem~\ref{thm:ub-prefix-length})
%
\item For intuition, we state a version of the above bound that is valid
at the start of execution. (Corollary~\ref{cor:ub-prefix-length})
%
\item By considering specific families of prefixes,
we can obtain tighter bounds on prefix length. (Theorem~\ref{thm:ub-prefix-specific})
\end{itemize}
In the next section (\S\ref{sec:ub-size}), we use these results
to derive corresponding upper bounds on the number of
prefix evaluations made by Algorithm~\ref{alg:branch-and-bound}.

\begin{proposition}[Trivial upper bound on prefix length]
\label{prop:trivial-length}
Consider a state space of all rule lists formed from
a set of~$M$ antecedents,
and let~$L(\RL)$ be the length of rule list~$\RL$.
%
$M$ provides an upper bound on the length of
any optimal rule list
${\OptimalRL \in \argmin_\RL \Obj(\RL, \x, \y)}$,
\ie ${L(\RL) \le M}$.
\end{proposition}

\begin{proof}
Rule lists consist of distinct rules by definition.
\end{proof}
\end{arxiv}

At any point during branch-and-bound execution, the current best objective~$\CurrentObj$
implies an upper bound on the maximum prefix length we might still have to consider.
%
\begin{theorem}[Upper bound on prefix length]
\label{thm:ub-prefix-length}
Consider a state space of all rule lists formed from a set of~$M$ antecedents.
%
Let~$L(\RL)$ be the length of rule list~$\RL$
and let~$\CurrentObj$ be the current best objective.
%
For all optimal rule lists ${\OptimalRL \in \argmin_\RL \Obj(\RL, \x, \y)}$
\begin{arxiv}
\begin{align}
L(\OptimalRL) \le \min \left(\left\lfloor \frac{\CurrentObj}{\Reg} \right\rfloor, M \right),
\label{eq:max-length}
\end{align}
\end{arxiv}
\begin{kdd}
\begin{align}
L(\OptimalRL) \le \min \left(\left\lfloor \CurrentObj / \Reg \right\rfloor, M \right),
\label{eq:max-length}
\end{align}
\end{kdd}
where~$\Reg$ is the regularization parameter.
%
\begin{arxiv}
Furthermore, if~$\CurrentRL$ is a rule list with
objective ${\Obj(\CurrentRL, \x, \y) = \CurrentObj}$,
length~$K$, and zero misclassification error,
then for every optimal rule list
${\OptimalRL \in}$ ${\argmin_\RL \Obj(\RL, \x, \y)}$,
if ${\CurrentRL \in \argmin_d \Obj(\RL, \x, \y)}$,
then ${L(\OptimalRL) \le K}$,
or otherwise if ${\CurrentRL \notin \argmin_d \Obj(\RL, \x, \y)}$,
then ${L(\OptimalRL) \le K - 1}$.
\end{arxiv}
\end{theorem}

\begin{arxiv}
\begin{proof}
For an optimal rule list~$\OptimalRL$ with objective~$\OptimalObj$,
\begin{align}
\Reg L(\OptimalRL) \le \OptimalObj = \Obj(\OptimalRL, \x, \y)
= \Loss(\OptimalRL, \x, \y) + \Reg L(\OptimalRL)
\le \CurrentObj. \nn
\end{align}
The maximum possible length for~$\OptimalRL$ occurs
when~$\Loss(\OptimalRL, \x, \y)$ is minimized;
combining with Proposition~\ref{prop:trivial-length}
gives bound~\eqref{eq:max-length}.

For the rest of the proof,
let~${K^* = L(\OptimalRL)}$ be the length of~$\OptimalRL$.
%
If the current best rule list~$\CurrentRL$ has zero
misclassification error, then
\begin{align}
\Reg K^* \leq \Loss(\OptimalRL, \x, \y) + \Reg K^* = \Obj(\OptimalRL, \x, \y)
\le \CurrentObj = \Obj(\CurrentRL, \x, \y) = \Reg K, \nn
\end{align}
and thus ${K^* \leq K}$.
%
If the current best rule list is suboptimal,
\ie ${\CurrentRL \notin \argmin_\RL \Obj(\RL, \x, \y)}$, then
%
\begin{align}
\Reg K^* \leq \Loss(\OptimalRL, \x, \y) + \Reg K^* = \Obj(\OptimalRL, \x, \y)
< \CurrentObj = \Obj(\CurrentRL, \x, \y) = \Reg K, \nn
\end{align}
in which case ${K^* < K}$, \ie ${K^* \leq K-1}$, since $K$ is an integer.
\end{proof}

The latter part of Theorem~\ref{thm:ub-prefix-length} tells us that
if we only need to identify a single instance of an optimal rule list
${\OptimalRL \in \argmin_\RL \Obj(\RL, \x, \y)}$, and we encounter a perfect
$K$-rule list with zero misclassification error, then we can prune all
prefixes of length~$K$ or greater.

\end{arxiv}

\begin{corollary}[Simple upper bound on prefix length]
\label{cor:ub-prefix-length}
\begin{arxiv}
Let~$L(\RL)$ be the length of rule list~$\RL$.
\end{arxiv}
%
For all optimal rule lists ${\OptimalRL \in \argmin_\RL \Obj(\RL, \x, \y)}$,
\begin{arxiv}
\begin{align}
L(\OptimalRL) \le \min \left( \left\lfloor \frac{1}{2\Reg} \right\rfloor, M \right).
\label{eq:max-length-trivial}
\end{align}
\end{arxiv}
\begin{kdd}
\begin{align}
L(\OptimalRL) \le \min \left( \left\lfloor 1 / 2\Reg \right\rfloor, M \right).
\label{eq:max-length-trivial}
\end{align}
\end{kdd}
\end{corollary}

\begin{arxiv}
\begin{proof}
Let ${d = ((), (), q_0, 0)}$ be the empty rule list;
it has objective ${\Obj(\RL, \x, \y) = \Loss(\RL, \x, \y) \le}$ ${1/2}$,
which gives an upper bound on~$\CurrentObj$.
%
Combining with~\eqref{eq:max-length}
and Proposition~\ref{prop:trivial-length}
gives~\eqref{eq:max-length-trivial}.
\end{proof}
\end{arxiv}

For any particular prefix~$\Prefix$, we can obtain potentially tighter
upper bounds on prefix length for
\begin{arxiv}
the family of
\end{arxiv}
all prefixes that start with~$\Prefix$.

\begin{theorem}[Prefix-specific upper bound on prefix length]
\label{thm:ub-prefix-specific}
Let ${\RL = (\Prefix, \Labels, \Default, K)}$ be a rule list, let
${\RL' = (\Prefix', \Labels', \Default', K') \in \StartsWith(\Prefix)}$
be any rule list such that~$\Prefix'$ starts with~$\Prefix$,
and let~$\CurrentObj$ be the current best objective.
%
If~$\Prefix'$ has lower bound~${b(\Prefix', \x, \y) < \CurrentObj}$, then
\begin{align}
K' < \min \left( K + \left\lfloor \frac{\CurrentObj - b(\Prefix, \x, \y)}{\Reg} \right\rfloor, M \right).
\label{eq:max-length-prefix}
\end{align}
\end{theorem}

\begin{arxiv}
\begin{proof}
First, note that~${K' \ge K}$, since~$\Prefix'$ starts with~$\Prefix$.
%
Now recall from~\eqref{eq:prefix-lb} that
%
\begin{align}
b(\Prefix, \x, \y) = \Loss_p(\Prefix, \Labels, \x, \y) + \Reg K
\le \Loss_p(\Prefix', \Labels', \x, \y) + \Reg K' = b(\Prefix', \x, \y), \nn
\end{align}
%
and from~\eqref{eq:prefix-loss} that
${\Loss_p(\Prefix, \Labels, \x, \y) \le \Loss_p(\Prefix', \Labels', \x, \y)}$.
%
Combining these bounds and rearranging gives
\begin{align}
b(\Prefix', \x, \y) &= \Loss_p(\Prefix', \Labels', \x, \y) + \Reg K + \Reg(K' - K) \nn \\
&\ge \Loss_p(\Prefix, \Labels, \x, \y) + \Reg K + \Reg(K' - K)
= b(\Prefix, \x, \y) + \Reg (K' - K).
\label{eq:length-diff}
\end{align}
Combining~\eqref{eq:length-diff} with~${b(\Prefix', \x, \y) < \CurrentObj}$
and Proposition~\ref{prop:trivial-length} gives~\eqref{eq:max-length-prefix}.
\end{proof}
\end{arxiv}

We can view Theorem~\ref{thm:ub-prefix-specific} as a generalization
of our one-step lookahead bound (Lemma~\ref{lemma:lookahead}),
as~\eqref{eq:max-length-prefix} is equivalently a bound on ${K' - K}$,
an upper bound on the number of remaining `steps' corresponding to
an iterative sequence of single-rule extensions of a prefix~$\Prefix$.
%
\begin{arxiv}
Notice that when~${\RL = ((), (), q_0, 0)}$ is the empty rule list,
this bound replicates~\eqref{eq:max-length}, since ${b(\Prefix, \x, \y) = 0}$.
\end{arxiv}

\begin{arxiv}
\subsection{Upper bounds on the number of prefix evaluations}
\end{arxiv}
\begin{kdd}
\subsection{Upper bounds on prefix evaluations}
\end{kdd}
\label{sec:ub-size}

\begin{arxiv}
In this section, we use our upper bounds on prefix length
from~\S\ref{sec:ub-prefix-length} to derive corresponding
upper bounds on the number of prefix evaluations made by
Algorithm~\ref{alg:branch-and-bound}.
%
First, we
\end{arxiv}
\begin{kdd}
In this section, we use Theorem~\ref{thm:ub-prefix-specific}'s
upper bound on prefix length to derive a corresponding
upper bound on the number of prefix evaluations made by
Algorithm~\ref{alg:branch-and-bound}.
%
We
\end{kdd}
present Theorem~\ref{thm:remaining-eval-fine},
in which we use information about the state of
Algorithm~\ref{alg:branch-and-bound}'s execution
to calculate, for any given execution state,
upper bounds on the number of additional prefix evaluations that might
be required for the execution to complete.
%
The relevant execution state depends on the current
best objective~$\CurrentObj$ and information about
prefixes we are planning to evaluate, \ie prefixes in the
queue~$\Queue$ of Algorithm~\ref{alg:branch-and-bound}.
%
We define the number of \emph{remaining prefix evaluations} as the number of
prefixes that are currently in or will be inserted into the queue.

\begin{arxiv}
We use Theorem~\ref{thm:remaining-eval-fine} in some of our empirical results
(\S\ref{sec:experiments}, Figure~\ref{fig:objective}) to help illustrate
the dramatic impact of certain algorithm optimizations.
%
The execution trace of this upper bound on remaining prefix evaluations
complements the execution traces of other quantities,
\eg that of the current best objective~$\CurrentObj$.
%
After presenting Theorem~\ref{thm:remaining-eval-fine}, we also give two
weaker propositions that provide useful intuition.
%
In particular, Proposition~\ref{prop:remaining-eval-coarse} is a practical
approximation to Theorem~\ref{thm:remaining-eval-fine} that is significantly
easier to compute; we use it in our implementation as a metric of
execution progress that we display to the user.
\end{arxiv}

\begin{arxiv}
\begin{theorem}[Fine-grained upper bound on remaining prefix evaluations]~
\end{arxiv}
\begin{kdd}
\begin{theorem}[Upper bound on the number of remaining prefix evaluations]
\end{kdd}
\label{thm:remaining-eval-fine}
~Consider the state space of all rule lists formed from a set of~$M$ antecedents,
and consider Algorithm~\ref{alg:branch-and-bound} at a particular instant
during execution.
%
Let~$\CurrentObj$ be the current best objective, let~$\Queue$ be the queue,
and let~$L(\Prefix)$ be the length of prefix~$\Prefix$.
%
Define~${\Remaining(\CurrentObj, \Queue)}$ to be the number of remaining
prefix evaluations, then
\begin{align}
\Remaining(\CurrentObj, \Queue)
\le \sum_{\Prefix \in Q} \sum_{k=0}^{f(\Prefix)} \frac{(M - L(\Prefix))!}{(M - L(\Prefix) - k)!},
\label{eq:remaining}
\end{align}
\begin{arxiv}
where
\begin{align}
f(\Prefix) = \min \left( \left\lfloor
  \frac{\CurrentObj - b(\Prefix, \x, \y)}{\Reg} \right\rfloor, M - L(\Prefix)\right). \nn
%\label{eq:f}
\end{align}
\end{arxiv}
\begin{kdd}
\begin{align}
\text{where} \quad f(\Prefix) = \min \left( \left\lfloor
  \frac{\CurrentObj - b(\Prefix, \x, \y)}{\Reg} \right\rfloor, M - L(\Prefix)\right).
%\label{eq:f}
\end{align}
\end{kdd}
\end{theorem}

\begin{proof}
The number of remaining prefix evaluations is equal to the number of
prefixes that are currently in or will be inserted into queue~$\Queue$.
%
For any such prefix~$\Prefix$, Theorem~\ref{thm:ub-prefix-specific}
gives an upper bound on the length of any prefix~$\Prefix'$
that starts with~$\Prefix$:
\begin{align}
L(\Prefix') \le \min \left( L(\Prefix) + \left\lfloor \frac{\CurrentObj - b(\Prefix, \x, \y)}{\Reg} \right\rfloor, M \right)
\equiv U(\Prefix). \nn
\end{align}
This gives an upper bound on the number of remaining prefix evaluations:
\begin{arxiv}
\begin{align}
\Remaining(\CurrentObj, \Queue)
\le \sum_{\Prefix \in Q} \sum_{k=0}^{U(\Prefix) - L(\Prefix)} P(M - L(\Prefix), k)
= \sum_{\Prefix \in Q} \sum_{k=0}^{f(\Prefix)} \frac{(M - L(\Prefix))!}{(M - L(\Prefix) - k)!}\,, \nn
\end{align}
where~$P(m, k)$ denotes the number of $k$-permutations of~$m$.
\end{arxiv}
\begin{kdd}
${\Remaining(\CurrentObj, \Queue)
\le \sum_{\Prefix \in Q} \sum_{k=0}^{U(\Prefix) - L(\Prefix)} P(M - L(\Prefix), k)}$.
\end{kdd}
\end{proof}

\begin{arxiv}
Proposition~\ref{thm:ub-total-eval} is strictly weaker than
Theorem~\ref{thm:remaining-eval-fine} and is the starting point for its derivation.
It
\end{arxiv}
\begin{kdd}
The proposition below
\end{kdd}
is a na\"ive upper bound on
the total number of prefix evaluations over the course of
Algorithm~\ref{alg:branch-and-bound}'s execution.
%
It only depends on the number of rules and
the regularization parameter~$\Reg$;
\ie unlike Theorem~\ref{thm:remaining-eval-fine},
it does not use algorithm execution state to
bound the size of the search space.

\begin{arxiv}
\begin{proposition}[Upper bound on the total number of prefix evaluations]
\end{arxiv}
\begin{kdd}
\begin{proposition}[Upper bound on the total number of prefix evaluations]
\end{kdd}
\label{thm:ub-total-eval}
~Define $\TotalRemaining(\RuleSet)$ to be the total number of prefixes
evaluated by Algorithm~\ref{alg:branch-and-bound}, given the state space of
all rule lists formed from a set~$\RuleSet$ of~$M$ rules.
%
For any set~$\RuleSet$ of $M$ rules,
\begin{arxiv}
\begin{align}
\TotalRemaining(\RuleSet) \le \sum_{k=0}^K \frac{M!}{(M - k)!}, \nn
%\label{eq:size-naive}
\end{align}
where ${K = \min(\lfloor 1/2 \Reg \rfloor, M)}$.
\end{arxiv}
\begin{kdd}
\begin{align}
\TotalRemaining(\RuleSet) \le \sum_{k=0}^K \frac{M!}{(M - k)!},
\quad \text{where} \quad K = \min(\lfloor 1/2 \Reg \rfloor, M).
\label{eq:size-naive}
\end{align}
\end{kdd}
\end{proposition}

\begin{proof}
By Corollary~\ref{cor:ub-prefix-length},
${K \equiv \min(\lfloor 1 / 2 \Reg \rfloor, M)}$
gives an upper bound on the length of any optimal rule list.
%
\begin{arxiv}
Since we can think of our problem as finding the optimal
selection and permutation of~$k$ out of~$M$ rules,
over all~${k \le K}$,
\begin{align}
\TotalRemaining(\RuleSet) \le 1 + \sum_{k=1}^K P(M, k)
= \sum_{k=0}^K \frac{M!}{(M - k)!}. \nn
\end{align}
\end{arxiv}
\begin{kdd}
We obtain~\eqref{eq:size-naive} by viewing
our problem as finding the optimal
selection and permutation of~$k$ out of~$M$ rules,
over all~${k \le K}$.
\end{kdd}
\end{proof}

\begin{arxiv}

Our next upper bound is strictly tighter than the bound in
Proposition~\ref{thm:ub-total-eval}.
%
Like Theorem~\ref{thm:remaining-eval-fine}, it uses the
current best objective and information about
the lengths of prefixes in the queue to constrain
the lengths of prefixes in the remaining search space.
%
However, Proposition~\ref{prop:remaining-eval-coarse}
is weaker than Theorem~\ref{thm:remaining-eval-fine} because
it leverages only coarse-grained information from the queue.
%
Specifically, Theorem~\ref{thm:remaining-eval-fine} is
strictly tighter because it additionally incorporates
prefix-specific objective lower bound information from
prefixes in the queue, which further constrains
the lengths of prefixes in the remaining search space.

\begin{proposition}[Coarse-grained upper bound on remaining prefix evaluations] \hfill
\label{prop:remaining-eval-coarse}
Consider a state space of all rule lists formed from a set of~$M$ antecedents,
and consider Algorithm~\ref{alg:branch-and-bound} at a particular instant
during execution.
%
Let~$\CurrentObj$ be the current best objective, let~$\Queue$ be the queue,
and let~$L(\Prefix)$ be the length of prefix~$\Prefix$.
%
Let~$\Queue_j$ be the number of prefixes of length~$j$ in~$\Queue$,
\begin{align}
\Queue_j = \big | \{ \Prefix : L(\Prefix) = j, \Prefix \in \Queue \} \big | \nn
\end{align}
and let~${J = \argmax_{\Prefix \in \Queue} L(\Prefix)}$
be the length of the longest prefix in~$\Queue$.
%
Define~${\Remaining(\CurrentObj, \Queue)}$ to be the number of remaining
prefix evaluations, then
\begin{align}
\Remaining(\CurrentObj, \Queue)
\le \sum_{j=1}^J \Queue_j \left( \sum_{k=0}^{K-j} \frac{(M-j)!}{(M-j - k)!} \right), \nn
\end{align}
where~${K = \min(\lfloor \CurrentObj / \Reg \rfloor, M)}$.
\end{proposition}

\begin{proof}
The number of remaining prefix evaluations is equal to the number of
prefixes that are currently in or will be inserted into queue~$\Queue$.
%
For any such remaining prefix~$\Prefix$,
Theorem~\ref{thm:ub-prefix-length} gives an upper bound on its length;
define~$K$ to be this bound:
${L(\Prefix) \le \min(\lfloor \CurrentObj / \Reg \rfloor, M) \equiv K}$.
%
For any prefix~$\Prefix$ in queue~$\Queue$ with length~${L(\Prefix) = j}$,
the maximum number of prefixes that start with~$\Prefix$
and remain to be evaluated is:
\begin{align}
\sum_{k=0}^{K-j} P(M-j, k) = \sum_{k=0}^{K-j} \frac{(M-j)!}{(M-j - k)!}, \nn
\end{align}
where~${P(T, k)}$ denotes the number of $k$-permutations of~$T$.
%
This gives an upper bound on the number of remaining prefix evaluations:
\begin{align}
\Remaining(\CurrentObj, \Queue)
\le \sum_{j=0}^J \Queue_j \left( \sum_{k=0}^{K-j} P(M-j, k) \right)
= \sum_{j=0}^J \Queue_j \left( \sum_{k=0}^{K-j} \frac{(M-j)!}{(M-j - k)!} \right). \nn
\end{align}
\end{proof}
\end{arxiv}

\subsection{Lower bounds on antecedent support}
\label{sec:lb-support}

In this section, we give two lower bounds on the normalized support
of each antecedent in any optimal rule list;
both are related to the regularization parameter~$\Reg$.

\begin{theorem}[Lower bound on antecedent support]
\label{thm:min-capture}
\begin{arxiv}
~Let ${\OptimalRL = (\Prefix, \Labels, \Default, K)}$
be any optimal rule list with objective~$\OptimalObj$, \ie
${\OptimalRL \in \argmin_\RL \Obj(\RL, \x, \y)}$.
\end{arxiv}
\begin{kdd}
Let ${\OptimalRL = (\Prefix, \Labels, \Default, K) \in \argmin_\RL \Obj(\RL, \x, \y)}$
be any optimal rule list, with objective~$\OptimalObj$.
\end{kdd}
For each antecedent~$p_k$ in prefix ${\Prefix = (p_1, \dots, p_K)}$,
\begin{arxiv}
the regularization parameter~$\Reg$ provides a lower bound
on the normalized support of~$p_k$,
\begin{align}
\Reg \le \Supp(p_k, \x \given \Prefix).
\label{eq:min-capture}
\end{align}
\end{arxiv}
\begin{kdd}
the regularization parameter provides a lower bound,
${\Reg \le \Supp(p_k, \x \given \Prefix)}$, on the normalized support of~$p_k$.
\end{kdd}
\end{theorem}

\begin{arxiv}
\begin{proof}
Let ${\OptimalRL = (\Prefix, \Labels, \Default, K)}$ be an optimal
rule list with prefix ${\Prefix = (p_1, \dots, p_K)}$
and labels ${\Labels = (q_1, \dots, q_K)}$.
%
Consider the rule list ${\RL = (\Prefix', \Labels', \Default', K-1)}$
derived from~$\OptimalRL$ by deleting a rule ${p_i \rightarrow q_i}$,
therefore ${\Prefix' = (p_1, \dots, p_{i-1}, p_{i+1}, \dots, p_K)}$
and ${\Labels' = (q_1, \dots, q_{i-1},}$ ${q'_{i+1}, \dots, q'_K)}$,
where~$q'_k$ need not be the same as~$q_k$, for ${k > i}$ and~${k = 0}$.

The largest possible discrepancy between~$\OptimalRL$ and~$\RL$ would occur
if~$\OptimalRL$ correctly classified all the data captured by~$p_i$,
while~$\RL$ misclassified these data.
%
This gives an upper bound:
\begin{align}
\Obj(\RL, \x, \y) = \Loss(\RL, \x, \y) + \Reg (K - 1)
&\le \Loss(\OptimalRL, \x, \y) + \Supp(p_i, \x \given \Prefix) + \Reg(K - 1) \nn \\
&= \Obj(\OptimalRL, \x, \y) + \Supp(p_i, \x \given \Prefix) - \Reg \nn \\
&= \OptimalObj + \Supp(p_i, \x \given \Prefix) - \Reg
\label{eq:ub-i}
\end{align}
where~$\Supp(p_i, \x \given \Prefix)$ is the normalized support of~$p_i$
in the context of~$\Prefix$, defined in~\eqref{eq:support-context},
and the regularization `bonus' comes from the fact that~$\RL$
is one rule shorter than~$\OptimalRL$.

At the same time, we must have ${\OptimalObj \le \Obj(\RL, \x, \y)}$ for~$\OptimalRL$ to be optimal.
%
Combining this with~\eqref{eq:ub-i} and rearranging gives~\eqref{eq:min-capture},
therefore the regularization parameter~$\Reg$ provides a lower bound
on the support of an antecedent~$p_i$ in an optimal rule list~$\OptimalRL$.
\end{proof}
\end{arxiv}

Thus, we can prune a prefix~$\Prefix$ if any of its antecedents captures less than
a fraction~$\Reg$ of data, even if~${b(\Prefix, \x, \y) < \OptimalObj}$.
%
\begin{arxiv}
Notice that the
\end{arxiv}
\begin{kdd}
The
\end{kdd}
bound in Theorem~\ref{thm:min-capture}
depends on the antecedents, but not the label predictions,
and thus does not account for misclassification error.
%
\begin{arxiv}
Theorem~\ref{thm:min-capture-correct} gives a tighter bound
by leveraging this additional information, which specifically
tightens the upper bound on~$\Obj(\RL, \x, \y)$ in~\eqref{eq:ub-i}.
\end{arxiv}
\begin{kdd}
Theorem~\ref{thm:min-capture-correct} gives a tighter bound
by leveraging this information.
\end{kdd}

\begin{theorem}[Lower bound on accurate antecedent support]
\label{thm:min-capture-correct}
\begin{arxiv}
Let ${\OptimalRL}$
be any optimal rule list with objective~$\OptimalObj$, \ie
${\OptimalRL = (\Prefix, \Labels, \Default, K) \in \argmin_\RL \Obj(\RL, \x, \y)}$.
%
Let $\OptimalRL$ have prefix ${\Prefix = (p_1, \dots, p_K)}$
and labels ${\Labels = (q_1, \dots, q_K)}$.
\end{arxiv}
\begin{kdd}
Let ${\OptimalRL \in \argmin_\RL \Obj(\RL, \x, \y)}$
be any optimal rule list, with objective~$\OptimalObj$;
let ${\OptimalRL = (\Prefix, \Labels, \Default, K)}$,
with prefix ${\Prefix = (p_1, \dots, p_K)}$
and labels ${\Labels = (q_1, \dots, q_K)}$.
\end{kdd}
%
For each rule~${p_k \rightarrow q_k}$ in~$\OptimalRL$,
define~$a_k$ to be the fraction of data that are captured by~$p_k$
and correctly classified:
\begin{align}
a_k \equiv \frac{1}{N} \sum_{n=1}^N
  \Cap(x_n, p_k \given \Prefix) \wedge \one [ q_k = y_n ].
\label{eq:rule-correct}
\end{align}
\begin{arxiv}
The regularization parameter~$\Reg$ provides a lower bound on~$a_k$:
\begin{align}
\Reg \le a_k.
\label{eq:min-capture-correct}
\end{align}
\end{arxiv}
\begin{kdd}
The regularization parameter provides a lower bound, $\Reg \le a_k$.
\end{kdd}
\end{theorem}

\begin{arxiv}
\begin{proof}
As in Theorem~\ref{thm:min-capture},
let ${\RL =  (\Prefix', \Labels', \Default', K-1)}$ be the rule list
derived from~$\OptimalRL$ by deleting a rule~${p_i \rightarrow q_i}$.
%
Now, let us define~$\Loss_i$ to be the portion of~$\OptimalObj$
due to this rule's misclassification error,
\begin{align}
\Loss_i \equiv \frac{1}{N} \sum_{n=1}^N
  \Cap(x_n, p_i \given \Prefix) \wedge \one [ q_i \neq y_n ]. \nn
\end{align}
The largest discrepancy between~$\OptimalRL$ and~$\RL$ would
occur if~$\RL$ misclassified all the data captured by~$p_i$.
%
This gives an upper bound on the difference between
the misclassification error of~$\RL$ and~$\OptimalRL$:
\begin{align}
\Loss(\RL, \x, \y) - \Loss(\OptimalRL, \x, \y)
&\le \Supp(p_i, \x \given \Prefix) - \Loss_i \nn \\
&= \frac{1}{N} \sum_{n=1}^N \Cap(x_n, p_i \given \Prefix)
  - \frac{1}{N} \sum_{n=1}^N
  \Cap(x_n, p_i \given \Prefix) \wedge \one [ q_i \neq y_n ] \nn \\
&= \frac{1}{N} \sum_{n=1}^N
  \Cap(x_n, p_i \given \Prefix) \wedge \one [ q_i = y_n ] = a_i, \nn
\end{align}
where we defined~$a_i$ in~\eqref{eq:rule-correct}.
%
Relating this bound to the objectives of~$\RL$ and~$\OptimalRL$ gives
\begin{align}
\Obj(\RL, \x, \y) = \Loss(\RL, \x, \y) + \Reg (K - 1)
&\le \Loss(\OptimalRL, \x, \y) + a_i + \Reg(K - 1) \nn \\
&= \Obj(\OptimalRL, \x, \y) + a_i - \Reg \nn \\
&= \OptimalObj + a_i - \Reg
\label{eq:ub-ii}
\end{align}
Combining~\eqref{eq:ub-ii} with the requirement
${\OptimalObj \le \Obj(\RL, \x, \y)}$ gives the bound~${\Reg \le a_i}$.
\end{proof}
\end{arxiv}

Thus, we can prune a prefix if any of its rules correctly classifies
less than a fraction~$\Reg$ of data.
%
While the lower bound in Theorem~\ref{thm:min-capture} is a sub-condition
of the lower bound in Theorem~\ref{thm:min-capture-correct},
we can still leverage both -- since the sub-condition is easier to check,
checking it first can accelerate pruning.
%
In addition to applying Theorem~\ref{thm:min-capture} in the context of
constructing rule lists, we can furthermore apply it in the context of
rule mining~(\S\ref{sec:setup}).
%
Specifically, it implies that we should only mine rules with
normalized support of at least~$\Reg$;
we need not mine rules with a smaller fraction of
\begin{arxiv}
observations.\footnote{We describe our application of this idea in
Appendix~\ref{appendix:data}, where we provide details on data processing.}
\end{arxiv}
\begin{kdd}
observations.\footnote{We describe our application of this idea in the appendix
of our long report~\citep{AngelinoLaAlSeRu17}.}
\end{kdd}
%
In contrast, we can only apply Theorem~\ref{thm:min-capture-correct}
in the context of constructing rule lists;
it depends on the misclassification error associated with each
rule in a rule list, thus it provides a lower bound on the number of
observations that each such rule must correctly classify.

\begin{arxiv}
\subsection{Upper bound on antecedent support}
\label{sec:ub-support}

In the previous section~(\S\ref{sec:lb-support}), we proved lower bounds on
antecedent support; in Appendix~\ref{appendix:ub-supp},
we give an upper bound on antecedent support.
%
Specifically, Theorem~\ref{thm:ub-support} shows that an antecedent's
support in a rule list cannot be too similar to the set of data not
captured by preceding antecedents in the rule list.
%
In particular, Theorem~\ref{thm:ub-support} implies that we should
only mine rules with normalized support less than or equal to ${1 - \Reg}$;
we need not mine rules with a larger fraction of observations.
%
Note that we do not otherwise use this bound in our implementation,
because we did not observe a meaningful benefit in preliminary experiments.
\end{arxiv}

\begin{arxiv}
\subsection{Antecedent rejection and its propagation}
\label{sec:reject}

In this section, we demonstrate further consequences of
our lower~(\S\ref{sec:lb-support}) and upper
bounds (\S\ref{sec:ub-support}) on antecedent support,
under a unified framework we refer to as antecedent rejection.
%
Let ${\Prefix = (p_1, \dots, p_K)}$ be a prefix,
and let~$p_k$ be an antecedent in~$\Prefix$.
%
Define~$p_k$ to have insufficient support in~$\Prefix$
if it does not obey the bound in~\eqref{eq:min-capture}
of Theorem~\ref{thm:min-capture}.
%
Define~$p_k$ to have insufficient accurate support in~$\Prefix$
if it does not obey the bound in~\eqref{eq:min-capture-correct}
of Theorem~\ref{thm:min-capture-correct}.
%
Define~$p_k$ to have excessive support in~$\Prefix$
if it does not obey the bound in~\eqref{eq:ub-support}
of Theorem~\ref{thm:ub-support} (Appendix~\ref{appendix:ub-supp}).
%
If~$p_k$ in the context of~$\Prefix$ has insufficient support,
insufficient accurate support, or excessive support,
%or support too similar to the set of data not captured by
%preceding antecedents (Theorem~\ref{thm:ub-support}),
let us say that prefix~$\Prefix$ rejects antecedent~$p_K$.
%
Next, in Theorem~\ref{thm:reject}, we describe large classes of
related rule lists whose prefixes all reject the same antecedent.

\begin{theorem}[Antecedent rejection propagates]
\label{thm:reject}
For any prefix ${\Prefix = (p_1, \dots, p_K)}$,
let $\StartContains(\Prefix)$ denote the set of all
prefixes~$\Prefix'$ such that
the set of all antecedents in~$\Prefix$ is a subset of
the set of all antecedents in~$\Prefix'$, \ie
\begin{align}
\StartContains(\Prefix) =
\{\Prefix' = (p'_1, \dots, p'_{K'})
~s.t.~ \{p_k : p_k \in \Prefix \} \subseteq
\{p'_\kappa : p'_\kappa \in \Prefix'\}, K' \ge K \}.
\label{eq:start-contains}
\end{align}
%
Let ${\RL = (\Prefix, \Labels, \Default, K)}$ be a rule list
with prefix ${\Prefix = (p_1, \dots, p_{K-1}, p_{K})}$,
such that~$\Prefix$ rejects its last antecedent~$p_{K}$,
either because~$p_{K}$ in the context of~$\Prefix$ has
insufficient support, insufficient accurate support,
or excessive support.
%
Let ${\Prefix^{K-1} = (p_1, \dots, p_{K-1})}$ be the
first~${K - 1}$ antecedents of~$\Prefix$.
%
Let ${\RLB = (\PrefixB, \LabelsB, \DefaultB, \kappa)}$
be any rule list with prefix
${\PrefixB = (P_1, \dots, P_{K'-1},}$ ${P_{K'}, \dots, P_{\kappa})}$
such that~$\PrefixB$ starts with ${\PrefixB^{K'-1} =}$
${(P_1, \dots, P_{K'-1}) \in}$ ${\StartContains(\Prefix^{K-1})}$
and antecedent ${P_{K'} = p_{K}}$.
%
It follows that prefix~$\PrefixB$ rejects~$P_{K'}$
for the same reason that~$\Prefix$ rejects~$p_{K}$,
and furthermore, $\RLB$~cannot be optimal, \ie
${\RLB \notin \argmin_{\RL^\dagger} \Obj(\RL^\dagger, \x, \y)}$.
\end{theorem}

\begin{proof}
Combine Proposition~\ref{prop:min-capture}, Proposition~\ref{prop:min-capture-correct},
and Proposition~\ref{prop:ub-support}.
%
The first two are found below, and the last in Appendix~\ref{appendix:ub-supp}.
\end{proof}

Theorem~\ref{thm:reject} implies potentially significant
computational savings.
%
We know from Theorems~\ref{thm:min-capture},
\ref{thm:min-capture-correct}, and~\ref{thm:ub-support}
that during branch-and-bound execution, if we ever encounter a
prefix ${\Prefix = (p_1, \dots, p_{K-1}, p_K)}$ that rejects its
last antecedent~$p_K$, then we can prune~$\Prefix$.
%
By Theorem~\ref{thm:reject}, we can also prune \emph{any} prefix~$\Prefix'$
whose antecedents contains the set of antecedents in~$\Prefix$,
in almost any order, with the constraint that all antecedents
in ${\{p_1, \dots, p_{K-1}\}}$ precede~$p_K$.
%
These latter antecedents are also rejected
directly by the bounds in Theorems~\ref{thm:min-capture},
\ref{thm:min-capture-correct}, and~\ref{thm:ub-support};
this is how our implementation works in practice.
%
In a preliminary implementation (not shown), we maintained additional
data structures to support the direct use of Theorem~\ref{thm:reject}.
%
We leave the design of efficient data structures for this task as future work.

\begin{proposition}[Insufficient antecedent support propagates]
\label{prop:min-capture}
First define~$\StartContains(\Prefix)$ as in~\eqref{eq:start-contains},
and let ${\Prefix = (p_1, \dots, p_{K-1}, p_{K})}$ be a prefix,
such that its last antecedent~$p_{K}$ has insufficient support,
\ie the opposite of the bound in~\eqref{eq:min-capture}:
${\Supp(p_K, \x \given \Prefix) < \Reg}$.
%
Let ${\Prefix^{K-1} =}$ ${(p_1, \dots, p_{K-1})}$,
and let ${\RLB = (\PrefixB, \LabelsB, \DefaultB, \kappa)}$
be any rule list with prefix
${\PrefixB = (P_1, \dots, P_{K'-1},}$ ${P_{K'}, \dots, P_{\kappa})}$,
such that~$\PrefixB$ starts with ${\PrefixB^{K'-1} =}$
${(P_1, \dots, P_{K'-1}) \in \StartContains(\Prefix^{K-1})}$
and~${P_{K'} = p_{K}}$.
%
It follows that~$P_{K'}$ has insufficient support in
prefix~$\PrefixB$, and furthermore, $\RLB$~cannot be optimal,
\ie ${\RLB \notin \argmin_{\RL} \Obj(\RL, \x, \y)}$.
\end{proposition}

\begin{proof}
The support of~$p_K$ in~$\Prefix$ depends only on the
set of antecedents in ${\Prefix^{K} = (p_1, \dots, p_{K})}$:
\begin{align}
\Supp(p_K, \x \given \Prefix)
= \frac{1}{N} \sum_{n=1}^N \Cap(x_n, p_K \given \Prefix)
&= \frac{1}{N} \sum_{n=1}^N \left( \neg\, \Cap(x_n, \Prefix^{K-1}) \right)
  \wedge \Cap(x_n, p_K) \nn \\
&= \frac{1}{N} \sum_{n=1}^N \left( \bigwedge_{k=1}^{K-1} \neg\, \Cap(x_n, p_k) \right)
  \wedge \Cap(x_n, p_K)
< \Reg, \nn
\end{align}
and the support of~$P_{K'}$ in~$\PrefixB$ depends only on
the set of antecedents in ${\PrefixB^{K'} =}$ ${(P_1, \dots, P_{K'})}$:
\begin{align}
\Supp(P_{K'}, \x \given \PrefixB)
= \frac{1}{N} \sum_{n=1}^N \Cap(x_n, P_{K'} \given \PrefixB) %\nn \\
%&= \frac{1}{N} \sum_{n=1}^N \left( \neg\, \Cap(x_n, \PrefixB^{K'-1}) \right)
%  \wedge \Cap(x_n, P_{K'}) \nn \\
&= \frac{1}{N} \sum_{n=1}^N \left( \bigwedge_{k=1}^{K'-1} \neg\, \Cap(x_n, P_k) \right)
   \wedge \Cap(x_n, P_{K'}) \nn \\
&\le \frac{1}{N} \sum_{n=1}^N \left( \bigwedge_{k=1}^{K-1} \neg\, \Cap(x_n, p_k) \right)
  \wedge \Cap(x_n, P_{K'}) \nn \\
&= \frac{1}{N} \sum_{n=1}^N \left( \bigwedge_{k=1}^{K-1} \neg\, \Cap(x_n, p_k) \right)
  \wedge \Cap(x_n, p_{K}) \nn \\
&= \Supp(p_K, \x \given \Prefix) < \Reg.
\label{ineq:supp}
\end{align}
The first inequality reflects the condition that
${\PrefixB^{K'-1} \in \StartContains(\Prefix^{K-1})}$,
which implies that the set of antecedents in~$\PrefixB^{K'-1}$
contains the set of antecedents in~$\Prefix^{K-1}$,
and the next equality reflects the fact that~${P_{K'} = p_K}$.
%
Thus,~$P_K'$ has insufficient support in prefix~$\PrefixB$,
therefore by Theorem~\ref{thm:min-capture}, $\RLB$~cannot be optimal,
\ie ${\RLB \notin \argmin_{\RL} \Obj(\RL, \x, \y)}$.
\end{proof}

\begin{proposition}[Insufficient accurate antecedent support propagates]
\label{prop:min-capture-correct}
~Let~$\StartContains(\Prefix)$ denote the set of all
prefixes~$\Prefix'$ such that
the set of all antecedents in~$\Prefix$ is a subset of
the set of all antecedents in~$\Prefix'$,
as in~\eqref{eq:start-contains}.
%
Let ${\RL = (\Prefix, \Labels, \Default, K)}$ be a rule list
with prefix ${\Prefix = (p_1, \dots, p_{K})}$
and labels ${\Labels = (q_1, \dots, q_{K})}$, such that
the last antecedent~$p_{K}$ has insufficient accurate support,
\ie the opposite of the bound in~\eqref{eq:min-capture-correct}:
\begin{align}
\frac{1}{N} \sum_{n=1}^N \Cap(x_n, p_K \given \Prefix) \wedge \one [ q_K = y_n ]
< \Reg. \nn
\end{align}
%
Let ${\Prefix^{K-1} = (p_1, \dots, p_{K-1})}$
and let ${\RLB = (\PrefixB, \LabelsB, \DefaultB, \kappa)}$
be any rule list with prefix ${\PrefixB =}$ ${(P_1, \dots, P_{\kappa})}$
and labels ${\LabelsB = (Q_1, \dots, Q_{\kappa})}$,
such that~$\PrefixB$ starts with ${\PrefixB^{K'-1} =}$
${(P_1, \dots, P_{K'-1})}$ ${\in \StartContains(\Prefix^{K-1})}$
and ${P_{K'} = p_{K}}$.
%
It follows that~$P_{K'}$ has insufficient accurate support in
prefix~$\PrefixB$, and furthermore,
${\RLB \notin \argmin_{\RL^\dagger} \Obj(\RL^\dagger, \x, \y)}$.
\end{proposition}

\begin{proof}
The accurate support of~$P_{K'}$ in~$\PrefixB$ is insufficient:
\begin{align}
\frac{1}{N} \sum_{n=1}^N \Cap(x_n, P_{K'} &\given \PrefixB) \wedge \one [ Q_{K'} = y_n ] \nn \\
&= \frac{1}{N} \sum_{n=1}^N \left( \bigwedge_{k=1}^{K'-1} \neg\, \Cap(x_n, P_k) \right)
   \wedge \Cap(x_n, P_{K'}) \wedge \one [ Q_{K'} = y_n ] \nn \\
&\le \frac{1}{N} \sum_{n=1}^N \left( \bigwedge_{k=1}^{K-1} \neg\, \Cap(x_n, p_k) \right)
   \wedge \Cap(x_n, P_{K'}) \wedge \one [ Q_{K'} = y_n ] \nn \\
&= \frac{1}{N} \sum_{n=1}^N \left( \bigwedge_{k=1}^{K-1} \neg\, \Cap(x_n, p_k) \right)
   \wedge \Cap(x_n, p_K) \wedge \one [ Q_{K'} = y_n ] \nn \\
&= \frac{1}{N} \sum_{n=1}^N \Cap(x_n, p_K \given \Prefix) \wedge \one [ Q_{K'} = y_n ] \nn \\
&\le \frac{1}{N} \sum_{n=1}^N \Cap(x_n, p_K \given \Prefix) \wedge \one [ q_{K} = y_n ]
< \Reg. \nn
\end{align}
The first inequality reflects the condition that
${\PrefixB^{K'-1} \in \StartContains(\Prefix^{K-1})}$,
the next equality reflects the fact that~${P_{K'} = p_K}$.
%
For the following equality, notice that~$Q_{K'}$ is the majority
class label of data captured by~$P_{K'}$ in~$\PrefixB$, and~$q_K$
is the majority class label of data captured by~$P_K$ in~$\Prefix$,
and recall from~\eqref{ineq:supp} that
${\Supp(P_{K'}, \x \given \PrefixB) \le \Supp(p_{K}, \x \given \Prefix)}$.
%
By Theorem~\ref{thm:min-capture-correct},
${\RLB \notin \argmin_{\RL^\dagger} \Obj(\RL^\dagger, \x, \y)}$.
\end{proof}

Propositions~\ref{prop:min-capture} and~\ref{prop:min-capture-correct},
combined with Proposition~\ref{prop:ub-support} (Appendix~\ref{appendix:ub-supp}),
constitute the proof of Theorem~\ref{thm:reject}.
\end{arxiv}

\subsection{Equivalent support bound}
\label{sec:equivalent}

If two prefixes capture the same data, and one is more accurate than the other,
then there is no benefit to considering prefixes that start with the less accurate one.
%
Let~$\Prefix$ be a prefix,
and consider the best possible rule list whose prefix starts with~$\Prefix$.
%
If we take its antecedents in~$\Prefix$ and replace them with another prefix
with the same support (that could include different antecedents),
then its objective can only become worse or remain the same.

Formally, let~$\PrefixB$ be a prefix, and let~$\xi(\PrefixB)$ be the set
of all prefixes that capture exactly the same data as~$\PrefixB$.
%
Now, let~$\RL$ be a rule list with prefix~$\Prefix$
in~$\xi(\PrefixB)$, such that~$\RL$ has the minimum objective
over all rule lists with prefixes in~$\xi(\PrefixB)$.
%
Finally, let~$\RL'$ be a rule list whose prefix~$\Prefix'$
starts with~$\Prefix$, such that~$\RL'$ has the minimum objective
over all rule lists whose prefixes start with~$\Prefix$.
%
Theorem~\ref{thm:equivalent} below implies that~$\RL'$ also has
the minimum objective over all rule lists whose prefixes start with
\emph{any} prefix in~$\xi(\PrefixB)$.

\begin{theorem}[Equivalent support bound]
\label{thm:equivalent}
Define $\StartsWith(\Prefix)$ to be the set of all rule lists
whose prefixes start with~$\Prefix$, as in~\eqref{eq:starts-with}.
%
Let ${\RL = (\Prefix, \Labels, \Default, K)}$
be a rule list with prefix ${\Prefix = (p_1, \dots, p_K)}$,
and let ${\RLB = (\PrefixB, \LabelsB, \DefaultB, \kappa)}$
be a rule list with prefix ${\PrefixB = (P_1, \dots, P_{\kappa})}$,
such that~$\Prefix$ and~$\PrefixB$ capture the same data,~\ie
\begin{align}
\{x_n : \Cap(x_n, \Prefix)\} = \{x_n : \Cap(x_n, \PrefixB)\}. \nn
\end{align}
%
If the objective lower bounds of~$\RL$ and~$\RLB$
obey ${b(\Prefix, \x, \y) \le b(\PrefixB, \x, \y)}$,
then the objective of the optimal rule list in~$\StartsWith(\Prefix)$ gives a
lower bound on the objective of the optimal rule list in~$\StartsWith(\PrefixB)$:
\begin{align}
\min_{\RL' \in \StartsWith(\Prefix)} \Obj(\RL', \x, \y)
\le \min_{\RLB' \in \StartsWith(\PrefixB)} \Obj(\RLB', \x, \y).
\label{eq:equivalent}
\end{align}
\end{theorem}

\begin{arxiv}
\begin{proof}
See Appendix~\ref{appendix:equiv-supp} for the proof of Theorem~\ref{thm:equivalent}.
\end{proof}
\end{arxiv}

Thus, if prefixes~$\Prefix$ and~$\PrefixB$ capture the same data,
and their objective lower bounds obey
${b(\Prefix, \x, \y) \le b(\PrefixB, \x, \y)}$,
Theorem~\ref{thm:equivalent} implies that we can prune~$\PrefixB$.
%
%In our implementation, we call this symmetry-aware garbage collection.
%
Next, in Sections~\ref{sec:permutation} and~\ref{sec:permutation-counting},
we highlight and analyze the special case of prefixes that capture
the same data because they contain the same antecedents.

\subsection{Permutation bound}% for permutation-aware garbage collection}
\label{sec:permutation}

If two prefixes are composed of the same antecedents,
\ie they contain the same antecedents up to a permutation,
then they capture the same data, and thus Theorem~\ref{thm:equivalent} applies.
%
Therefore, if one is more accurate than the other, then there is no benefit to
considering prefixes that start with the less accurate one.
%
Let~$\Prefix$ be a prefix,
and consider the best possible rule list whose prefix starts with~$\Prefix$.
%
If we permute its antecedents in~$\Prefix$,
then its objective can only become worse or remain the same.

Formally, let~${P = \{p_k\}_{k=1}^K}$ be a set of~$K$ antecedents,
and let~$\Pi$ be the set of all $K$-prefixes corresponding to
permutations of antecedents in~$P$.
%
Let prefix~$\Prefix$ in~$\Pi$ have the minimum prefix misclassification
error over all prefixes in~$\Pi$.
%
Also, let~$\RL'$ be a rule list whose prefix~$\Prefix'$
starts with~$\Prefix$, such that~$\RL'$ has the minimum objective
over all rule lists whose prefixes start with~$\Prefix$.
%
Corollary~\ref{thm:permutation} below,
which can be viewed as special case of Theorem~\ref{thm:equivalent},
implies that~$\RL'$ also has the minimum objective over all
rule lists whose prefixes start with \emph{any} prefix in~$\Pi$.

\begin{corollary}[Permutation bound]
\label{thm:permutation}
Let~$\pi$ be any permutation of ${\{1, \dots, K\}}$,
\begin{arxiv}
and define ${\StartsWith(\Prefix) = }$
${\{(\Prefix', \Labels', \Default', K') : \Prefix' \textnormal{ starts with } \Prefix \}}$
to be the set of all rule lists whose prefixes start with~$\Prefix$.
\end{arxiv}
\begin{kdd}
and define $\StartsWith(\Prefix)$
to be the set of all rule lists whose prefix starts with~$\Prefix$,
as in~\eqref{eq:starts-with}.
\end{kdd}
%
Let ${\RL = (\Prefix, \Labels, \Default, K)}$
and ${\RLB = (\PrefixB, \LabelsB, \DefaultB, K)}$
denote rule lists with prefixes ${\Prefix = (p_1, \dots, p_K)}$
and ${\PrefixB = (p_{\pi(1)}, \dots, p_{\pi(K)})}$,
respectively, \ie the antecedents in~$\PrefixB$
correspond to a permutation of the antecedents in~$\Prefix$.
%
If the objective lower bounds of~$\RL$ and~$\RLB$
obey ${b(\Prefix, \x, \y) \le b(\PrefixB, \x, \y)}$,
then the objective of the optimal rule list in~$\StartsWith(\Prefix)$ gives a
lower bound on the objective of the optimal rule list in~$\StartsWith(\PrefixB)$:
\begin{align}
\min_{\RL' \in \StartsWith(\Prefix)} \Obj(\RL', \x, \y)
\le \min_{\RLB' \in \StartsWith(\PrefixB)} \Obj(\RLB', \x, \y). \nn
%\label{eq:permutation}
\end{align}
\end{corollary}

\begin{arxiv}
\begin{proof}
Since prefixes~$\Prefix$ and~$\PrefixB$ contain
the same antecedents, they both capture the same data.
Thus, we can apply Theorem~\ref{thm:equivalent}.
\end{proof}
\end{arxiv}

Thus if prefixes~$\Prefix$ and~$\PrefixB$ have the same antecedents,
up to a permutation, and their objective lower bounds
obey~${b(\Prefix, \x, \y) \le}$ ${b(\PrefixB, \x, \y)}$,
Corollary~\ref{thm:permutation} implies that we can prune~$\PrefixB$.
%
We call this %permutation-aware garbage collection,
symmetry-aware pruning,
and we illustrate the subsequent
computational savings next in~\S\ref{sec:permutation-counting}.

\begin{arxiv}
\subsection{Upper bound on prefix evaluations with symmetry-aware pruning}
\end{arxiv}
\begin{kdd}
\subsubsection{Upper bound on prefix evaluations with symmetry-aware pruning}
\end{kdd}
\label{sec:permutation-counting}

Here, we present an upper bound on the total number of prefix
evaluations that accounts for the effect of symmetry-aware
pruning~(\S\ref{sec:permutation}).
%
Since every subset of~$K$ antecedents generates an equivalence
class of~$K!$ prefixes equivalent up to permutation, symmetry-aware
pruning dramatically reduces the search space.

\begin{arxiv}
First, notice that
\end{arxiv}
Algorithm~\ref{alg:branch-and-bound} describes a
breadth-first exploration of the state space of rule lists.
%
Now suppose we integrate symmetry-aware pruning into
our execution of branch-and-bound, so that after evaluating
prefixes of length~$K$, we only keep a single best prefix
from each set of prefixes equivalent up to a permutation.

\begin{theorem}[Upper bound on prefix evaluations with symmetry-aware pruning]
%
Consider a state space of all rule lists formed from a set~$\RuleSet$
of~$M$ antecedents, and consider the branch-and-bound algorithm with
symmetry-aware pruning.
%
Define $\TotalRemaining(\RuleSet)$ to be the total number of prefixes evaluated.
%
For any set~$\RuleSet$ of $M$ rules,
\begin{align}
\TotalRemaining(\RuleSet)
\le  1 + \sum_{k=1}^K \frac{1}{(k - 1)!} \cdot \frac{M!}{(M - k)!}, \nn
\end{align}
where ${K = \min(\lfloor 1 / 2 \Reg \rfloor, M)}$.
\end{theorem}

\begin{proof}
By Corollary~\ref{cor:ub-prefix-length},
${K \equiv \min(\lfloor 1 / 2 \Reg \rfloor, M)}$
gives an upper bound on the length of any optimal rule list.
%
The algorithm begins by evaluating the empty prefix,
followed by~$M$ prefixes of length~${k=1}$,
then~${P(M, 2)}$ prefixes of length~${k=2}$,
where~${P(M, 2)}$ is the number of size-2 subsets of~$\{1, \dots, M \}$.
%
Before proceeding to length~${k=3}$, we keep only~${C(M, 2)}$
prefixes of length~${k=2}$, where~${C(M, k)}$ denotes the
number of $k$-combinations of~$M$.
%
Now, the number of length~${k=3}$ prefixes we evaluate is~${C(M, 2) (M - 2)}$.
%
Propagating this forward~gives
\begin{arxiv}
\begin{align}
\TotalRemaining(\RuleSet) \le 1 + \sum_{k=1}^K C(M, k-1) (M - k + 1)
%= 1 + \sum_{k=1}^K {M \choose k-1}(M - k + 1)
%= 1 + \sum_{k=1}^K \frac{M! (M - k + 1)}{(k - 1)! (M - k + 1)!}
= 1 + \sum_{k=1}^K \frac{1}{(k - 1)!} \cdot \frac{M!}{(M - k)!}. \nn
\end{align}
\end{arxiv}
\begin{kdd}
${\TotalRemaining(\RuleSet) \le 1 + \sum_{k=1}^K C(M, k-1) (M - k + 1)}$.
\end{kdd}
\end{proof}

Pruning based on permutation symmetries thus yields significant
computational savings.
%
Let us compare, for example, to the na\"ive number of prefix evaluations
given by the upper bound in Proposition~\ref{thm:ub-total-eval}.
%
If~${M = 100}$ and~${K = 5}$, then the na\"ive number is about
${9.1 \times 10^9}$, while the reduced number due to symmetry-aware
pruning is about ${3.9 \times 10^8}$,
which is smaller by a factor of about~23.
%
If~${M=1000}$ and~${K = 10}$, the number of evaluations falls from
about~${9.6 \times 10^{29}}$ to about~${2.7 \times 10^{24}}$,
which is smaller by a factor of about~360,000.
\begin{arxiv}

\end{arxiv}
\begin{kdd}
%
\end{kdd}
While~$10^{24}$ seems infeasibly enormous,
it does not represent the number of rule lists we evaluate.
%
\begin{kdd}
As we show in~\S\ref{sec:experiments},
\end{kdd}
\begin{arxiv}
As we show in our experiments~(\S\ref{sec:experiments}),
\end{arxiv}
our permutation bound in Corollary~\ref{thm:permutation}
and our other bounds together conspire to reduce the search space
to a size manageable on a single computer.
%
The choice of ${M=1000}$ and ${K=10}$ in our example above
corresponds to the state space size our efforts target.
%
${K=10}$ rules represents a (heuristic) upper limit on
the size of an interpretable rule list,
and ${M=1000}$ represents the approximate number of rules
with sufficiently high support (Theorem~\ref{thm:min-capture})
we expect to obtain via rule mining~(\S\ref{sec:setup}).

\begin{arxiv}
\subsection{Similar support bound}
\label{sec:similar}

We now present a relaxation of Theorem~\ref{thm:equivalent},
our equivalent support bound.
%
Theorem~\ref{thm:similar} implies that if we know that no extensions of
a prefix~$\Prefix$ are better than the current best objective,
then we can prune all prefixes with support similar to~$\Prefix$'s support.
%
Understanding how to exploit this result in practice
represents an exciting direction for future work;
our implementation~(\S\ref{sec:implementation}) does not
currently leverage the bound in Theorem~\ref{thm:similar}.

\begin{theorem}[Similar support bound]
\label{thm:similar}
Define $\StartsWith(\Prefix)$ to be the set of all rule lists
whose prefixes start with~$\Prefix$, as in~\eqref{eq:starts-with}.
%
Let ${\Prefix = (p_1, \dots, p_K)}$ and
${\PrefixB = (P_1, \dots, P_{\kappa})}$ be prefixes
that capture nearly the same data.
%
Specifically, define~$\omega$ to be the normalized support
of data captured by~$\Prefix$ and not captured by~$\PrefixB$, \ie
%and let us require that~${\omega \le \Reg}$, \ie
\begin{align}
\omega \equiv \frac{1}{N} \sum_{n=1}^N
  \neg\, \Cap(x_n, \PrefixB)
  \wedge \Cap(x_n, \Prefix). % \le \Reg,
\label{eq:omega}
\end{align}
%where~$\Reg$ is the regularization parameter.
%
Similarly, define~$\Omega$ to be the normalized support
of data captured by~$\PrefixB$ and not captured by~$\Prefix$, \ie
%and let us require that~${\Omega \le \Reg}$, \ie
\begin{align}
\Omega \equiv \frac{1}{N} \sum_{n=1}^N
  \neg\, \Cap(x_n, \Prefix)
  \wedge \Cap(x_n, \PrefixB). %\le \Reg.
\label{eq:big-omega}
\end{align}
We can bound the difference between the objectives of the
optimal rule lists in~$\StartsWith(\Prefix)$
and $\StartsWith(\PrefixB)$ as follows:
\begin{align}
\min_{\RLB^\dagger \in \StartsWith(\PrefixB)} \Obj(\RLB^\dagger, \x, \y)
- \min_{\RL^\dagger \in \StartsWith(\Prefix)} \Obj(\RL^\dagger, \x, \y)
&\ge b(\PrefixB, \x, \y) - b(\Prefix, \x, \y) - \omega - \Omega,
\label{eq:similar}
\end{align}
where~$b(\Prefix, \x, \y)$ and~$b(\PrefixB, \x, \y)$ are the
objective lower bounds of~$\RL$ and~$\RLB$, respectively.
\end{theorem}

\begin{proof}
See Appendix~\ref{appendix:similar-supp} for the proof of Theorem~\ref{thm:similar}.
\end{proof}

Theorem~\ref{thm:similar} implies that if prefixes~$\Prefix$
and~$\PrefixB$ are similar, and we know the optimal objective
of rule lists starting with~$\Prefix$, then
\begin{align}
\min_{\RLB' \in \StartsWith(\PrefixB)} \Obj(\RLB', \x, \y)
&\ge \min_{\RL' \in \StartsWith(\Prefix)} \Obj(\RL', \x, \y)
+ b(\PrefixB, \x, \y) - b(\Prefix, \x, \y) - \chi \nn \\
&\ge \CurrentObj + b(\PrefixB, \x, \y) - b(\Prefix, \x, \y) - \chi, \nn
\end{align}
where~$\CurrentObj$ is the current best objective,
and~$\chi$ is the normalized support of the set of data captured
either exclusively by~$\Prefix$ or exclusively by~$\PrefixB$.
%
It follows that
\begin{align}
\min_{\RLB' \in \StartsWith(\PrefixB)} \Obj(\RLB', \x, \y)
\ge \CurrentObj + b(\PrefixB, \x, \y) - b(\Prefix, \x, \y) - \chi \ge \CurrentObj \nn
\end{align}
if ${b(\PrefixB, \x, \y) - b(\Prefix, \x, \y) \ge \chi}$.
%
To conclude, we summarize this result and combine it with
our notion of lookahead from Lemma~\ref{lemma:lookahead}.
%
During branch-and-bound execution, if we demonstrate that
${\min_{\RL' \in \StartsWith(\Prefix)} \Obj(\RL', \x, \y) \ge \CurrentObj}$,
then we can prune all prefixes that start with any
prefix~$\PrefixB'$ in the following set:
\begin{align}
\left\{ \PrefixB' : b(\PrefixB', \x, \y) + \Reg - b(\Prefix, \x, \y) \ge
\frac{1}{N} \sum_{n=1}^N \Cap(x_n, \Prefix) \oplus \Cap(x_n, \PrefixB') \right\}, \nn
\end{align}
where the symbol~$\oplus$ denotes the logical operation, exclusive or (XOR).

\begin{comment}
\begin{theorem}[Very similar support bound]
\label{thm:very-similar}
Define ${\StartsWith(\Prefix) = }$
${\{(\Prefix', \Labels', \Default', K') : \Prefix' \textnormal{ starts with } \Prefix \}}$
to be the set of all rule lists whose prefixes start with~$\Prefix$.
%
Let ${\Prefix = (p_1, \dots, p_K)}$ and
${\PrefixB = (P_1, \dots, P_{\kappa})}$ be prefixes
that capture the same data, \ie
\begin{align}
\{x_n : \Cap(x_n, \Prefix)\} = \{x_n : \Cap(x_n, \PrefixB)\}.
\end{align}
Now let ${\Prefix' = (p_1, \dots, p_K, p_{K+1})}$ be a prefix
that starts with~$\Prefix$ and ends with antecedent~$p_{K+1}$,
and let ${\PrefixB' = (P_1, \dots, P_{\kappa}, P_{\kappa + 1})}$
be a prefix that starts with~$\PrefixB$ and ends with
antecedent~$P_{\kappa + 1}$, such that~$p_{K+1}$
and~$P_{\kappa + 1}$ have nearly the same support in~$\Prefix'$
and~$\PrefixB'$, respectively.
%
Specifically, define~$\omega$ to be the normalized support
of data captured by~$p_{K + 1}$ in~$\Prefix'$ and not
captured by~$P_{\kappa + 1}$ in~$\PrefixB'$,
and let us require that~${\omega \le \Reg}$, \ie
\begin{align}
\omega \equiv \frac{1}{N} \sum_{n=1}^N
  \neg\, \Cap(x_n, P_{\kappa+1} \given \PrefixB')
  \wedge \Cap(x_n, p_{K+1} \given \Prefix') \le \Reg,
\end{align}
where~$\Reg$ is the regularization parameter.
%
Similarly, define~$\Omega$ to be the normalized support
of data captured by~$P_{\kappa + 1}$ in~$\PrefixB'$ and not
captured by~$p_{K + 1}$ in~$\Prefix'$,
and let us require that~${\Omega \le \Reg}$, \ie
\begin{align}
\Omega \equiv \frac{1}{N} \sum_{n=1}^N
  \neg\, \Cap(x_n, p_{K+1} \given \Prefix')
  \wedge \Cap(x_n, P_{\kappa+1} \given \PrefixB') \le \Reg.
\end{align}
The difference between the objective of the optimal rule list
in~$\StartsWith(\Prefix')$ and that in~$\StartsWith(\PrefixB')$
is at most~$2\Reg$:
\begin{align}
\left | \min_{\RL^\dagger \in \StartsWith(\Prefix')} \Obj(\RL^\dagger, \x, \y)
  - \min_{\RLB^\dagger \in \StartsWith(\PrefixB')} \Obj(\RLB^\dagger, \x, \y)
  \right | \le 2 \Reg.
\label{eq:very-similar}
\end{align}
\end{theorem}

\begin{proof}
\dots
\end{proof}

Suppose prefixes~$\cal P$ and~$\cal Q$ capture the same data,
and now derive~$\cal P'$ from~$\cal P$ by appending antecedent~$p$
and derive~$\cal Q'$ from~$\cal Q$ by appending antecedent~$q$.
%
Suppose further that~$p$ and~$q$ capture nearly the same data, except that
they exclusively capture data~$x_p$ and~$x_q$ in their respective contexts,
such that the normalized support of~$x_p$ and of~$x_q$ are each bounded by
the regularization parameter: ${s(x_p), s(x_q) < c}$.
%
Our minimum support bound~\eqref{eq:min-capture} implies
that~$p$ would never be placed below~$\cal Q'$ nor~$q$ below~$\cal P'$.

Extensions of~$\cal P'$ and~$\cal Q'$ will behave similarly.
%
The largest difference would occur if a rule list starting with~$\cal P'$
misclassified all of~$x_p$ and~$x_q$, while the analogous rule list starting
with~$\cal Q'$ correctly classified all these data, or vice versa,
yielding a difference between objectives bounded by~$2c$.
%
Let~$\cal P^*$ and~$\cal Q^*$ be the optimal prefixes
starting with~$\cal P'$ and~$\cal Q'$, respectively.
%
Note that~$\cal P^*$ and~$\cal Q^*$ need not be derived from~$\cal P'$ and~$\cal Q'$
via analogous extensions.
%
If we know~$\cal P^*$, then we can avoid evaluating \emph{any} extensions of~$\cal Q'$ if
\begin{align}
\Obj(\Prefix^*) - 2 c \ge \Obj^*,
\end{align}
where~$\Obj^*$ is the best known objective, since the left had expression
provides a lower bound on~$\Obj(\cal{Q}^*)$.
\end{comment}

\end{arxiv}

\subsection{Equivalent points bound}
\label{sec:identical}

The bounds in this section quantify the following:
%
If multiple observations that are not captured by a prefix~$\Prefix$
have identical features and opposite labels, then no rule list that
starts with~$\Prefix$ can correctly classify all these observations.
%
For each set of such observations, the number of mistakes is at least
the number of observations with the minority label within the set.

Consider a data set~${\{(x_n, y_n)\}_{n=1}^N}$ and also a set of antecedents
${\{s_m\}_{m=1}^M}$.
%
Define distinct observations to be equivalent if they are captured by
exactly the same antecedents, \ie ${x_i \neq x_j}$ are equivalent~if
\begin{align}
\frac{1}{M} \sum_{m=1}^M \one [ \Cap(x_i, s_m) = \Cap(x_j, s_m) ] = 1. \nn
\end{align}
Notice that we can partition a data set into sets of equivalent points;
let~${\{e_u\}_{u=1}^U}$ enumerate these sets.
%
Let~$e_u$ be the equivalent points set that contains observation~$x_i$.
%
Now define~$\theta(e_u)$ to be the normalized support of the minority
class label with respect to set~$e_u$, \eg let
\begin{arxiv}
\begin{align}
{e_u = \{x_n : \forall m \in [M],\, \one [ \Cap(x_n, s_m) = \Cap(x_i, s_m) ] \}}, \nn
\end{align}
\end{arxiv}
\begin{kdd}
${e_u = \{x_n : \forall m \in [M],\, \one [ \Cap(x_n, s_m) = \Cap(x_i, s_m) ] \}}$,
\end{kdd}
and let~$q_u$ be the minority class label among points in~$e_u$, then
\begin{align}
\theta(e_u) = \frac{1}{N} \sum_{n=1}^N \one [ x_n \in e_u ]\, \one [ y_n = q_u ].
\label{eq:theta}
\end{align}

The existence of equivalent points sets with non-singleton support
yields a tighter objective lower bound that we can combine with our other bounds;
as our experiments demonstrate~(\S\ref{sec:experiments}),
the practical consequences can be dramatic.
%
First, for intuition, we present a general bound in
Proposition~\ref{prop:identical}; next, we explicitly integrate
this bound into our framework in Theorem~\ref{thm:identical}.

\begin{proposition}[General equivalent points bound]
\label{prop:identical}
Let ${\RL = (\Prefix, \Labels, \Default, K)}$ be a rule list, then
\begin{arxiv}
\begin{align}
\Obj(\RL, \x, \y) \ge \sum_{u=1}^U \theta(e_u) + \Reg K. \nn
\end{align}
\end{arxiv}
\begin{kdd}
${\Obj(\RL, \x, \y) \ge \sum_{u=1}^U \theta(e_u) + \Reg K}$.
\end{kdd}
\end{proposition}

\begin{arxiv}
\begin{proof}
Recall that the objective is ${\Obj(\RL, \x, \y) = \Loss(\RL, \x, \y) + \Reg K}$,
where the misclassification error~${\Loss(\RL, \x, \y)}$ is given by
\begin{align}
\Loss(\RL, \x, \y) &= \Loss_0(\Prefix, \Default, \x, \y) + \Loss_p(\Prefix, \Labels, \x, \y) \nn \\
&= \frac{1}{N} \sum_{n=1}^N \left( \neg\, \Cap(x_n, \Prefix) \wedge \one[q_0 \neq y_n]
   + \sum_{k=1}^K \Cap(x_n, p_k \given \Prefix) \wedge \one [q_k \neq y_n] \right). \nn
\end{align}
Any particular rule list uses a specific rule, and therefore a single class label,
to classify all points within a set of equivalent points.
%
Thus, for a set of equivalent points~$u$, the rule list~$\RL$ correctly classifies either
points that have the majority class label, or points that have the minority class label.
%
It follows that~$\RL$ misclassifies a number of points in~$u$ at least as great as
the number of points with the minority class label.
%
To translate this into a lower bound on~$\Loss(\RL, \x, \y)$,
we first sum over all sets of equivalent points, and then for each such set,
count differences between class labels and the minority class label of the set,
instead of counting mistakes:
\begin{align}
&\Loss(\RL, \x, \y) \nn \\
&= \frac{1}{N} \sum_{u=1}^U \sum_{n=1}^N \left( \neg\, \Cap(x_n, \Prefix) \wedge \one[q_0 \neq y_n]
   + \sum_{k=1}^K \Cap(x_n, p_k \given \Prefix) \wedge \one [q_k \neq y_n] \right)
   \one [x_n \in e_u]  \nn \\
&\ge \frac{1}{N} \sum_{u=1}^U \sum_{n=1}^N \left( \neg\, \Cap(x_n, \Prefix) \wedge \one[y_n = q_u]
   + \sum_{k=1}^K \Cap(x_n, p_k \given \Prefix) \wedge \one [y_n = q_u] \right)
   \one [x_n \in e_u].
\label{eq:lb-equiv-pts}
\end{align}
Next, we factor out the indicator for equivalent point set membership,
which yields a term that sums to one, because every datum is either captured or
not captured by prefix~$\Prefix$.
\begin{align}
\Loss(\RL, \x, \y) &= \frac{1}{N} \sum_{u=1}^U \sum_{n=1}^N \left( \neg\, \Cap(x_n, \Prefix)
   + \sum_{k=1}^K \Cap(x_n, p_k \given \Prefix) \right)
   \wedge \one [x_n \in e_u]\, \one[y_n = q_u] \nn \\
&= \frac{1}{N} \sum_{u=1}^U \sum_{n=1}^N \left( \neg\, \Cap(x_n, \Prefix)
   + \Cap(x_n, \Prefix) \right)
   \wedge \one [x_n \in e_u]\, \one[y_n = q_u] \nn \\
&= \frac{1}{N} \sum_{u=1}^U \sum_{n=1}^N \one [ x_n \in e_u ]\, \one [ y_n = q_u ]
= \sum_{u=1}^U \theta(e_u), \nn
\end{align}
where the final equality applies the definition of~$\theta(e_u)$ in~\eqref{eq:theta}.
%
Therefore, ${\Obj(\RL, \x, \y) =}$ ${\Loss(\RL, \x, \y) + \Reg K}$ ${\ge \sum_{u=1}^U \theta(e_u) + \Reg K}$.
\end{proof}
\end{arxiv}

Now, recall that to obtain our lower bound~${b(\Prefix, \x, \y)}$
in~\eqref{eq:lower-bound}, we simply deleted the
default rule misclassification error~$\Loss_0(\Prefix, \Default, \x, \y)$
from the objective~${\Obj(\RL, \x, \y)}$.
%
Theorem~\ref{thm:identical} obtains a tighter objective lower bound
via a tighter lower bound on the default rule misclassification error,
${0 \le b_0(\Prefix, \x, \y) \le}$ $\Loss_0(\Prefix, \Default, \x, \y)$.

\begin{theorem}[Equivalent points bound]
\label{thm:identical}
Let~$\RL$ be a rule list with prefix~$\Prefix$
and lower bound ${b(\Prefix, \x, \y)}$,
then for any rule list~${\RL' \in \StartsWith(\RL)}$
whose prefix~$\Prefix'$ starts with~$\Prefix$,
\begin{arxiv}
\begin{align}
\Obj(\RL', \x, \y) \ge b(\Prefix, \x, \y) + b_0(\Prefix, \x, \y),
\label{eq:identical}
\end{align}
where
\end{arxiv}
\begin{kdd}
\begin{align}
\Obj(\RL', \x, \y) \ge b(\Prefix, \x, \y) + b_0(\Prefix, \x, \y), \quad \text{where}
\end{align}
\end{kdd}
\begin{arxiv}
\begin{align}
b_0(\Prefix, \x, \y) = \frac{1}{N} \sum_{u=1}^U \sum_{n=1}^N
    \neg\, \Cap(x_n, \Prefix) \wedge \one [ x_n \in e_u ]\, \one [ y_n = q_u ].
\label{eq:lb-b0}
\end{align}
\end{arxiv}
\begin{kdd}
\begin{align}
b_0(\Prefix, \x, \y) = \frac{1}{N} \sum_{u=1}^U \sum_{n=1}^N
    \neg\, \Cap(x_n, \Prefix) \wedge \one [ x_n \in e_u ]\, \one [ y_n = q_u ]. \nn
\end{align}
\end{kdd}
\end{theorem}

\begin{arxiv}
\begin{proof}
See Appendix~\ref{appendix:equiv-pts} for the proof of Theorem~\ref{thm:identical}.
\end{proof}
\end{arxiv}
