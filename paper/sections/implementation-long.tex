\begin{arxiv}
\section{Incremental computation}
\label{sec:incremental}
\end{arxiv}

For every prefix~$\Prefix$ evaluated during
Algorithm~\ref{alg:branch-and-bound}'s execution, we compute
the objective lower bound~${b(\Prefix, \x, \y)}$ and sometimes
the objective~${\Obj(\RL, \x, \y)}$ of the corresponding rule list~$\RL$.
%
These calculations are the dominant
\begin{arxiv}
computations with respect to execution time.
%
This motivates
\end{arxiv}
\begin{kdd}
computations, and motivate
\end{kdd}
our use of a highly optimized library,
designed by~\citet{YangRuSe16} for representing rule lists and
performing operations encountered in evaluating functions of rule lists.
%
Furthermore, we exploit the hierarchical nature of the objective
function and its lower bound to compute these quantities
incrementally throughout branch-and-bound execution.
%
\begin{arxiv}
In this section, we provide explicit expressions for
the incremental computations that are central to our approach.
%
Later, in~\S\ref{sec:implementation}, we describe a cache data structure
for supporting our incremental framework in practice.

For completeness, before presenting our incremental expressions,
let us begin by writing down the objective lower bound and objective
of the empty rule list, ${\RL = ((), (), \Default, 0)}$,
the first rule list evaluated in Algorithm~\ref{alg:branch-and-bound}.
%
Since its prefix contains zero rules, it has zero prefix
misclassification error and also has length zero.
%
Thus, the empty rule list's objective lower bound is zero, \ie ${b((), \x, \y) = 0}$.
%
Since none of the data are captured by the empty prefix, the default rule
corresponds to the majority class, and the objective corresponds to the
default rule misclassification error, \ie ${\Obj(\RL, \x, \y) = \Loss_0((), \Default, \x, \y)}$.

Now, we derive our incremental expressions for the objective function and its lower bound.
%
Let ${\RL = (\Prefix, \Labels, \Default, K)}$ and
${\RL' = (\Prefix', \Labels', \Default', K + 1)}$
be rule lists such that prefix ${\Prefix =}$ ${(p_1, \dots, p_K)}$
is the parent of ${\Prefix' = (p_1, \dots, p_K, p_{K+1})}$.
%
Let ${\Labels = (q_1, \dots, q_K)}$ and
${\Labels' = (q_1, \dots,}$ ${q_K, q_{K+1})}$ be the corresponding labels.
%
The hierarchical structure of Algorithm~\ref{alg:branch-and-bound}
enforces that if we ever evaluate~$\RL'$, then we will have already
evaluated both the objective and objective lower bound of its parent,~$\RL$.
%
We would like to reuse as much of these computations as possible
in our evaluation of~$\RL'$.
%
We can write the objective lower bound of~$\RL'$ incrementally,
with respect to the objective lower bound of~$\RL$:
\begin{align}
b(\Prefix', \x, \y)
  &= \Loss_p(\Prefix', \Labels', \x, \y) + \Reg (K + 1) \nn \\
&= \frac{1}{N} \sum_{n=1}^N \sum_{k=1}^{K+1} \Cap(x_n, p_k \given \Prefix')
  \wedge \one [ q_k \neq y_n ] + \Reg (K+1) \label{eq:non-inc-lb} \\
&= \Loss_p(\Prefix, \Labels, \x, \y) + \Reg K + \Reg
  + \frac{1}{N} \sum_{n=1}^N \Cap(x_n, p_{K+1} \given \Prefix') \wedge \one [q_{K+1} \neq y_n ] \nn \\
&= b(\Prefix, \x, \y) + \Reg
  + \frac{1}{N} \sum_{n=1}^N \Cap(x_n, p_{K+1} \given \Prefix') \wedge \one [q_{K+1} \neq y_n ] \nn \\
&= b(\Prefix, \x, \y) + \Reg  + \frac{1}{N} \sum_{n=1}^N \neg\, \Cap(x_n, \Prefix) \wedge
  \Cap(x_n, p_{K+1}) \wedge \one [q_{K+1} \neq y_n].
\label{eq:inc-lb}
\end{align}
%Notice that the term~${\neg\, \Cap(x_n, \Prefix)}$ in~\eqref{eq:inc-lb}
%depends only on~$\Prefix$, \ie it does not depend on~$\Prefix'$; furthermore,
%it is computed in the evaluation of~$\RL$'s default rule misclassification error,
%\begin{align}
%\Loss_0(\Prefix, \Default, \x, \y) = \frac{1}{N}\sum_{n=1}^N \neg\, \Cap(x_n, \Prefix) \wedge \one [q_0 \neq y_n].
%\end{align}
Thus, if we store $b(\Prefix, \x, \y)$, % and ${\neg\, \Cap(\x, \Prefix)}$,
then we can reuse this quantity when computing $b(\Prefix', \x, \y)$.
%
Transforming~\eqref{eq:non-inc-lb} into~\eqref{eq:inc-lb} yields a
significantly simpler expression that is a function of the stored
quantity~$b(\Prefix, \x, \y)$. %, as well as ${\neg\, \Cap(\x, \Prefix)}$
%and the last rule of~$\RL'$, ${p_{K+1} \rightarrow q_{K+1}}$.
%
For the objective of~$\RL'$, first let us write a na\"ive expression:
\begin{align}
&\Obj(\RL', \x, \y) = \Loss(\RL', \x, \y) + \Reg (K + 1)
= \Loss_p(\Prefix', \Labels', \x, \y) + \Loss_0(\Prefix', \Default', \x, \y) + \Reg(K + 1) \nn \\
&= \frac{1}{N} \sum_{n=1}^N \sum_{k=1}^{K+1} \Cap(x_n, p_k \given \Prefix')
  \wedge \one [ q_k \neq y_n ] + \frac{1}{N}\sum_{n=1}^N \neg\, \Cap(x_n, \Prefix') \wedge
  \one [q'_0 \neq y_n] + \Reg (K+1). \label{eq:non-inc-obj}
\end{align}
Instead, we can compute the objective of~$\RL'$ incrementally
with respect to its objective lower bound:
\begin{align}
\Obj(\RL', \x, \y) &=  \Loss_p(\Prefix', \Labels', \x, \y) +
  \Loss_0(\Prefix', \Default', \x, \y) + \Reg (K + 1) \nn \\
&= b(\Prefix', \x, \y) + \Loss_0(\Prefix', \Default', \x, y) \nn \\
&= b(\Prefix', \x, \y) + \frac{1}{N}\sum_{n=1}^N \neg\, \Cap(x_n, \Prefix') \wedge
  \one [q'_0 \neq y_n] \nn \\
&= b(\Prefix', \x, \y) + \frac{1}{N}\sum_{n=1}^N \neg\, \Cap(x_n, \Prefix) \wedge
  (\neg\, \Cap(x_n, p_{K+1})) \wedge \one [q'_0 \neq y_n].
\label{eq:inc-obj}
\end{align}
The expression in~\eqref{eq:inc-obj} is simpler to compute than that
in~\eqref{eq:non-inc-obj}, because the former reuses $b(\Prefix', \x, \y)$,
which we already computed in~\eqref{eq:inc-lb}.
%as well as ${\neg\, \Cap(\x, \Prefix)}$,
%and the last antecedent~$p_{K+1}$ and default rule~$\Default'$ of~$\RL'$.
Note that instead of computing~$\Obj(\RL', \x, \y)$ incrementally from $b(\Prefix', \x, \y)$
as in~\eqref{eq:inc-obj}, we could have computed it incrementally from $\Obj(\RL, \x, \y)$.
However, doing so would in practice require that we store~$\Obj(\RL, \x, \y)$
in addition to~$b(\Prefix, \x, \y)$, which we already must store to support~\eqref{eq:inc-lb}.
We prefer the incremental approach suggested by~\eqref{eq:inc-obj}
since it avoids this additional storage overhead.

\begin{algorithm}[t!]
  \caption{Incremental branch-and-bound for learning rule lists, for simplicity, from a cold start.
  We explicitly show the incremental objective lower bound and objective functions in Algorithms~\ref{alg:incremental-lb} and~\ref{alg:incremental-obj}, respectively.}
\label{alg:incremental}
\begin{algorithmic}
\normalsize
\State \textbf{Input:} Objective function~$\Obj(\RL, \x, \y)$,
objective lower bound~${b(\Prefix, \x, \y)}$,
set of antecedents ${\RuleSet = \{s_m\}_{m=1}^M}$,
training data~$(\x, \y) = {\{(x_n, y_n)\}_{n=1}^N}$,
regularization parameter~$\Reg$
\State \textbf{Output:} Provably optimal rule list~$\OptimalRL$ with minimum objective~$\OptimalObj$ \\

\State $\CurrentRL \gets ((), (), \Default, 0)$ \Comment{Initialize current best rule list with empty rule list}
\State $\CurrentObj \gets \Obj(\CurrentRL, \x, \y)$ \Comment{Initialize current best objective}
\State $Q \gets $ queue$(\,[\,(\,)\,]\,)$ \Comment{Initialize queue with empty prefix}
\State $C \gets $ cache$(\,[\,(\,(\,)\,, 0\,)\,]\,)$ \Comment{Initialize cache with empty prefix and its objective lower bound}
\While {$Q$ not empty} \Comment{Optimization complete when the queue is empty}
	\State $\Prefix \gets Q$.pop(\,) \Comment{Remove a length-$K$ prefix~$\Prefix$ from the queue}
        \State $b(\Prefix, \x, \y) \gets C$.find$(\Prefix)$ \Comment{Look up $\Prefix$'s lower bound in the cache}
        \State $\mathbf{u} \gets \neg\,\Cap(\x, \Prefix)$ \Comment{Bit vector indicating data not captured by $\Prefix$}
        \For {$s$ in $\RuleSet$} \Comment{Evaluate all of $\Prefix$'s children}
            \If {$s$ not in $\Prefix$}
                \State $\PrefixB \gets (\Prefix, s)$ \Comment{\textbf{Branch}: Generate child $\PrefixB$}
                \State $\mathbf{v} \gets \mathbf{u} \wedge \Cap(\x, s)$ \Comment{Bit vector indicating data captured by $s$ in $\PrefixB$}
                \State $b(\PrefixB, \x, \y) \gets b(\Prefix, \x, \y) + \Reg~ + $ \Call{IncrementalLowerBound}{$\mathbf{v}, \y, N$} %\Comment{Eq.~\eqref{eq:inc-lb}}
                \If {$b(\PrefixB, \x, \y) < \CurrentObj$} \Comment{\textbf{Bound}: Apply bound from Theorem~\ref{thm:bound}}
                    \State $\Obj(\RLB, \x, \y) \gets b(\PrefixB, \x, \y)~ + $ \Call{IncrementalObjective}{$\mathbf{u}, \mathbf{v}, \y, N$} %\Comment{Eq.~\eqref{eq:inc-obj}}
                    \If {$\Obj(\RLB, \x, \y) < \CurrentObj$}
                        \State $\RLB \gets (\PrefixB, \LabelsB, \DefaultB, K+1)$ \Comment{$\LabelsB, \DefaultB$ are set in the incremental functions}
                        \State $(\CurrentRL, \CurrentObj) \gets (\RLB, \Obj(\RLB, \x, \y))$ \Comment{Update current best rule list and objective}
                    \EndIf
                    \State $Q$.push$(\PrefixB)$ \Comment{Add $\PrefixB$ to the queue}
                    \State $C$.insert$(\PrefixB, b(\PrefixB, \x, \y))$ \Comment{Add $\PrefixB$ and its lower bound to the cache}
                \EndIf
            \EndIf
        \EndFor
\EndWhile
\State $(\OptimalRL, \OptimalObj) \gets (\CurrentRL, \CurrentObj)$ \Comment{Identify provably optimal rule list and objective}
\end{algorithmic}
\end{algorithm}

\begin{algorithm}[t!]
  \caption{Incremental objective lower bound~\eqref{eq:inc-lb} used in Algorithm~\ref{alg:incremental}.}
\label{alg:incremental-lb}
\begin{algorithmic}
\normalsize
\State \textbf{Input:}
Bit vector~${\mathbf{v} \in \{0, 1\}^N}$ indicating data captured by $s$, the last antecedent in~$\PrefixB$,
bit vector of class labels~${\y \in \{0, 1\}^N}$,
number of observations~$N$
\State \textbf{Output:} Component of~$\RLB$'s misclassification error due to data captured by~$s$ \\

\Function{IncrementalLowerBound}{$\mathbf{v}, \y, N$}
    \State $n_v = \Count(\mathbf{v})$ \Comment{Number of data captured by $s$, the last antecedent in $\PrefixB$}
    \State $\mathbf{w} \gets \mathbf{v} \wedge \y$ \Comment{Bit vector indicating data captured by $s$ with label $1$}
    \State $n_w = \Count(\mathbf{w})$ \Comment{Number of data captured by $s$ with label $1$}
    \If {$n_w / n_v > 0.5$}
        \State \Return $(n_v - n_w) / N$ \Comment{Misclassification error of the rule $s \rightarrow 1$}
    \Else
        \State \Return $n_w / N$ \Comment{Misclassification error of the rule $s \rightarrow 0$}
    \EndIf
    \EndFunction
\end{algorithmic}
\end{algorithm}

\begin{algorithm}[t!]
  \caption{Incremental objective function~\eqref{eq:inc-obj} used in Algorithm~\ref{alg:incremental}.}
\label{alg:incremental-obj}
\begin{algorithmic}
\normalsize
\State \textbf{Input:}
Bit vector~${\mathbf{u} \in \{0, 1\}^N}$ indicating data not captured by~$\PrefixB$'s parent prefix,
bit vector~${\mathbf{v} \in \{0, 1\}^N}$ indicating data not captured by $s$, the last antecedent in~$\PrefixB$,
bit vector of class labels~${\y \in \{0, 1\}^N}$,
number of observations~$N$
\State \textbf{Output:} Component of~$\RLB$'s misclassification error due to its default rule \\

 \Function{IncrementalObjective}{$\mathbf{u}, \mathbf{v}, \y, N$}
    \State $\mathbf{f} \gets \mathbf{u} \wedge \neg\,\mathbf{v} $ \Comment{Bit vector indicating data not captured by $\PrefixB$}
    \State $n_f = \Count(\mathbf{f})$ \Comment{Number of data not captured by $\PrefixB$}
    \State $\mathbf{g} \gets \mathbf{f} \wedge \y$ \Comment{Bit vector indicating data not captured by $\PrefixB$ with label $1$}
    \State $n_g = \Count(\mathbf{g})$ \Comment{Number of data not captued by $\PrefixB$ with label $1$}
    \If {$n_g / n_f > 0.5$}
        \State \Return $(n_f - n_g) / N$ \Comment{Misclassification error of the default label prediction $1$}
    \Else
        \State \Return $n_g / N$ \Comment{Misclassification error of the default label prediction $0$}
    \EndIf
\EndFunction
\end{algorithmic}
\end{algorithm}

We present an incremental branch-and-bound procedure in
Algorithm~\ref{alg:incremental}, and show the incremental computations
of the objective lower bound~\eqref{eq:inc-lb} and objective~\eqref{eq:inc-obj}
as two separate functions in Algorithms~\ref{alg:incremental-lb}
and~\ref{alg:incremental-obj}, respectively.
%
In Algorithm~\ref{alg:incremental}, we use a cache to store
prefixes and their objective lower bounds.
%
Algorithm~\ref{alg:incremental} additionally reorganizes the structure
of Algorithm~\ref{alg:branch-and-bound} to group together the computations
associated with all children of a particular prefix.
%
This has two advantages.
%
The first is to consolidate cache queries: all children of the same
parent prefix compute their objective lower bounds with respect to
the parent's stored value, and we only require one cache `find' operation
for the entire group of children, instead of a separate query for each child.
%
The second is to shrink the queue's size:
instead of adding all of a prefix's children as separate queue elements,
we represent the entire group of children in the queue by a single element.
%
Since the number of children associated with each prefix
is close to the total number of possible antecedents,
both of these effects can yield significant savings.
%
For example, if we are trying to optimize over rule lists formed
from a set of 1000 antecedents, then the maximum queue size in
Algorithm~\ref{alg:incremental} will be smaller than that in
Algorithm~\ref{alg:branch-and-bound} by a factor of nearly 1000.

\end{arxiv}


\section{Implementation}
\label{sec:implementation}

We implement our algorithm using a collection of optimized data structures.
%
First, in~\S\ref{sec:trie}, we explain our choice of a prefix tree
to support incremental computation~(\S\ref{sec:incremental}).
%
Second, in~\S\ref{sec:queue}, we describe several queue designs
that implement different search policies.
%
Third, in~\S\ref{sec:pmap}, we introduce a symmetry-aware map to support
symmetry-aware pruning~(Corollary~\ref{thm:permutation},~\S\ref{sec:permutation}).
%
Next, in~\S\ref{sec:execution}, we summarize our incremental execution model,
and in~\S\ref{sec:gc}, we describe how we garbage collect our data structures.
%
We additionally highlight in~\S\ref{sec:scheduling} how our queue can be used to support
custom scheduling policies designed to improve performance.
%
Our implementation of CORELS can be found~at: \\

\centerline{\url{https://github.com/nlarusstone/corels}.}

\subsection{Prefix tree}
\label{sec:trie}

Our incremental computations (\S\ref{sec:incremental}) require a
cache to keep track of rule lists we have already evaluated.
%
We implement this cache as a prefix tree, a data structure also known as a trie,
which allows us to efficiently represent structure shared between related rule lists.
%
Each node in the prefix tree encodes an individual rule ${r_k = p_k \rightarrow q_k}$.
%
Each path starting from the root represents a rule list, such that the final node
in the path also contains metadata associated with that corresponding rule list.
%
For a rule list ${\RL = (\Prefix, \Labels, \Default, K)}$,
with prefix ${\Prefix = (p_1, \dots, p_K)}$,
let~$\varphi(\RL)$ denote the corresponding node in the trie.
%
The metadata at node~$\varphi(\RL)$ supports the incremental computations
we described in~\S\ref{sec:incremental}, and includes:
\begin{itemize}
\item An index encoding antecedent~$p_K$.
\item The corresponding label prediction~$q_K$.
%\item $\RL$'s length~$K$; equivalently, node~$\varphi(\RL)$'s depth in the trie. % <-- not strictly necessary
\item The default rule label prediction~$\Default$.
\item $\NCap$, the number of samples captured by prefix $\Prefix$, as in~\eqref{eq:num-cap}. % <-- not strictly necessary
% \item The objective $\Obj(\RL, \x, \y)$~\eqref{eq:objective}. % <-- not strictly necessary
\item The objective lower bound $b(\Prefix, \x, \y)$, defined in~\eqref{eq:lower-bound},
  the central bound in our framework (Theorem~\ref{thm:bound}).
\item The lower bound on the default rule misclassification error
  $b_0(\Prefix, \x, \y)$, defined in~\eqref{eq:lb-b0},
  to support our equivalent points bound (Theorem~\ref{thm:identical}).
\item An indicator denoting whether or not this node should be deleted (see~\S\ref{sec:gc}).
\item A representation of viable extensions of~$\Prefix$,
  \ie length ${K+1}$ rule lists that start with~$\Prefix$ and have not been pruned.
\end{itemize}
We note that we implement the prefix tree as a custom C++ class. % decouple artifact from design
%
% This might be a bit much detail for here
%In addition to our base trie class, we also implemented a different node type that we use in our algorithm.
%This sub-class has an additional field that can hold custom metrics that we use to order the search space.
%Since this additional field is just a double, the memory overhead is minimal.
%
% Interesting trie-related subroutines besides garbage collection?

\subsection{Queue}
\label{sec:queue}

The queue is a worklist that orders exploration over the search space of possible
rule lists; every queue element corresponds to a prefix tree leaf, and vice versa.
%
In our implementation, each queue element points to a leaf;
when we pop an element off the queue, we use the leaf's metadata to
incrementally evaluate the corresponding prefix's children.

We order entries in the queue to implement several different policies.
%
A first-in-first-out (FIFO) queue implements breadth-first search (BFS),
and a priority queue implements best-first search; in our released code,
we implement all scheduling policies, including BFS, using a STL C++ priority queue.
%
Example priority queue policies include ordering
by the lower bound, the objective, a custom metric that maps prefixes to real values,
or prefix length, which corresponds to depth-first search (DFS).
%
As we demonstrate in our experiments~(\S\ref{sec:ablation}),
we find that using an custom search strategy,
such as ordering by the lower bound, usually leads to a faster runtime than BFS.

In preliminary work (not shown), we also experimented with
stochastic exploration processes that bypass the need for a queue
by instead following random paths from the root to leaves;
developing such methods could be an interesting direction for future work.

\subsection{Symmetry-aware map}
\label{sec:pmap}

The symmetry-aware map supports symmetry-aware pruning~(\S\ref{sec:permutation}).
%
In our implementation, we specifically leverage our permutation bound
(Corollary~\ref{thm:permutation}), though it is also possible to directly
exploit the more general equivalent support bound (Theorem~\ref{thm:equivalent}).
%
We implement this symmetry-aware map using the C++ STL unordered\_map,
to map all permutations of a set of antecedents to a key whose value contains
the best ordering of those antecedents (\ie the prefix with the smallest lower bound).
%
Every antecedent is associated with an index, and we call the numerically
sorted order of a set of antecedents its canonical order.
%
Thus by querying a set of antecedents by its canonical order,
all permutations map to the same key.
%
The value stored in the map consists of the lower bound and the actual ordering
of the rules that is best for that permutation.

Before we consider adding a prefix~$\Prefix$ to the trie and queue, we check
whether the map already contains a permutation~$\pi(\Prefix)$ of that prefix.
%
If no such permutation exists, then we insert~$\Prefix$ into the map, trie, and queue.
%
Otherwise, if a permutation~$\pi(\Prefix)$ exists and the lower bound of~$\Prefix$ is better
than that of~$\pi(\Prefix)$, \ie ${b(\Prefix, \x, \y) <}$ ${b(\pi(\Prefix), \x, \y)}$,
then we update the map and remove~$\pi(P_j)$ and its entire subtree from the trie;
we also insert~$\Prefix$ into the trie and queue.
%
Otherwise, if there exists a permutation~$\pi(\Prefix)$ such that
${b(\pi(\Prefix), \x, \y) \le}$ ${b(\Prefix, \x, \y)}$,
then we do nothing, \ie we do not insert~$\Prefix$ into any data structures.

\subsection{Incremental execution}
\label{sec:execution}

Mapping our algorithm to our data structures produces the following execution strategy.
%
We initialize the current best objective~$\CurrentObj$ and rule list~$\CurrentRL$.
%
While the trie contains unexplored leaves, a scheduling policy selects the next prefix~$\Prefix$
to extend; in our implementation, we pop elements from a (priority) queue, until the queue is empty.
%
Then, for every antecedent~$s$ that is not in~$\Prefix$,
we construct a new prefix~$\Prefix'$ by appending~$s$ to~$\Prefix$;
we incrementally calculate the lower bound~$b(\Prefix', \x, \y)$,
the objective~$\Obj(\RL', \x, \y)$, of the associated rule list~$\RL'$,
and other quantities used by our algorithm, summarized by the metadata fields of
the (potential) prefix tree node~$\varphi(\Prefix')$.

If the objective~$\Obj(\RL', \x, \y)$ is less than the current best objective~$\CurrentObj$,
then we update~$\CurrentObj$ and~$\RL$.
%
If the lower bound of the new prefix~$\Prefix'$ is less than the current best objective,
%\ie ${b(\Prefix', \x, \y) < \CurrentObj}$,
then as described in~\S\ref{sec:pmap}, we query the symmetry-aware map for~$\Prefix'$;
if we insert~$\Prefix'$ into the symmetry-aware map, then we also insert it into the trie and queue.
%
Otherwise, %\ie ${b(\Prefix', \x, \y) \ge \CurrentObj}$
then by our hierarchical lower bound (Theorem~\ref{thm:bound}),
no extension of~$\Prefix'$ could possibly lead to a rule list with objective
better than~$\CurrentObj$, thus we do not insert~$\Prefix'$ into the tree or queue.
%
We also leverage our other bounds from~\S\ref{sec:framework}
to aggressively prune the search space.
%
When there are no more leaves to explore, \ie the queue is empty, we output the optimal rule list.
%
We can also choose to terminate early according to an alternate condition,
\eg when the size of the prefix tree exceeds some threshold.

\subsection{Garbage collection}
\label{sec:gc}

During execution, we garbage collect the trie.
%
Each time we update the minimum objective,
we traverse the trie in a depth-first manner, deleting all subtrees
of any node with lower bound larger than the current minimum objective.
%
At other times, when we encounter a node with no children, we prune upwards,
deleting that node and recursively traversing the tree towards the root,
deleting any childless nodes.
%
This garbage collection allows us to constrain the trie's memory consumption, though in our
experiments we observe the minimum objective to decrease only a small number of times.

In our implementation, we cannot immediately garbage collect trie elements that are currently in the queue.
%
The STL C++ priority queue is a wrapper container that prevents access to the underlying data structure.
%
Therefore we cannot access elements in the middle of the queue,
even know the relevant identifying information.
%
Thus, we have no way to update the queue without iterating over every element.
%
We address this by lazily marking nodes in the prefix tree as deleted (see~\S\ref{sec:trie}),
without deleting the physical node until it has been removed from the queue.
%
We define two different queues that we refer to in our experiments~(\S\ref{sec:experiments}):
the physical queue corresponds to all elements in the C++ queue, and thus all prefix tree leaves,
and the logical queue corresponds only to those prefix tree leaves that have not been marked deleted.

\subsection{Custom scheduling policies}
\label{sec:scheduling}

In our setting, an ideal scheduling policy would immediately identify an optimal
rule list, and then certify its optimality by systematically eliminating the
remaining search space.
%
This motivates trying to design scheduling policies that tend to quickly find optimal rule lists.
%
When we use a priority queue to order the set of prefixes to evaluate next,
we are free to implement different scheduling policies via the ordering of
elements in the queue.
%
This motivates designing functions that assign higher priorities to `better'
prefixes that we believe are more likely to lead to optimal rule lists.
%
Note that we follow the convention that priority queue elements are ordered
by keys, such that keys with smaller values correspond to higher priorities.

We introduce a custom class of functions that we call \emph{curiosity} functions.
%
Broadly, we think of the curiosity of a rule list~$\RL$
as the expected objective value of another rule list~$\RL'$ that is related to~$\RL$;
different models of the relationship between~$\RL$ and~$\RL'$ lead to different
curiosity functions.
%
In general, the curiosity of~$\RL$ is by definition equal to the sum of the expected
misclassification error and the expected regularization penalty of~$\RL'$:
\begin{align}
\Curiosity(\Prefix, \x, \y) \equiv \E[ \Obj(\RL', \x, \y) ]
&= \E[\Loss(\Prefix', \Labels', \x, \y)] + \Reg \E[ K' ].
\label{eq:curiosity}
\end{align}

Next, we describe a simple curiosity function for a rule list~$\RL$ with prefix~$\Prefix$.
%
First, let~$\NCap$ denote the number of datapoints captured by~$\Prefix$, \ie
\begin{align}
\NCap \equiv \sum_{n=1}^N \Cap(x_n, \Prefix).
\label{eq:num-cap}
\end{align}
We now describe a model that generates another
rule list~${\RL' = (\Prefix', \Labels', \Default', K')}$ from~$\Prefix$.
%
Assume that prefix~$\Prefix'$ starts with~$\Prefix$ and captures all the data,
such that each additional antecedent in~$\Prefix'$
captures as many `new' datapoints as each antecedent in~$\Prefix$, on average;
then, the expected length of~$\Prefix'$ is:
\begin{align}
\E[ K' ] = \frac{N}{\NCap / K}.
\label{eq:curiosity-length}
\end{align}
Furthermore, assume that each additional antecedent in~$\Prefix'$
makes as many mistakes as each antecedent in~$\Prefix$, on average,
thus the expected misclassification error of~$\Prefix'$ is:
\begin{align}
\E[\Loss(\Prefix', \Labels', \x, \y)]
&= \E[\Loss_p(\Prefix', \Labels', \x, \y)] + \E[\Loss_0(\Prefix', \Default', \x, \y)] \nn \\
&= \E[\Loss_p(\Prefix', \Labels', \x, \y)]
=  \E[ K' ] \left(\frac{\Loss_p(\Prefix, \Labels, \x, \y)}{K}\right).
\label{eq:curiosity-error}
\end{align}
Note that the default rule misclassification error~$\Loss_0(\Prefix', \Default', \x, \y)$
is zero because we assume that~$\Prefix'$ captures all the data.
%
Combining~\eqref{eq:curiosity-length}~\eqref{eq:curiosity-error} and thus gives
curiosity for this model:
\begin{align*}
\Curiosity(\Prefix, \x, \y)
%= \E[\Loss_p(\Prefix', \Labels', \x, \y)] + \Reg \E[ K' ]
%&= \E[ K' ] \left(\frac{\Loss_p(\Prefix, \Labels, \x, \y)}{K}\right) + \Reg \E[ K' ] \\
%&=  \left(\frac{N}{\NCap / K}\right)
%  \left(\frac{\Loss_p(\Prefix, \Labels, \x, \y)}{K}\right)
%  + \Reg \left(\frac{N}{\NCap / K}\right) \nn \\
&= \left( \frac{N}{\NCap} \right) \biggl(\Loss_p(\Prefix, \Labels, \x, \y) + \Reg K \biggr) \\
&= \left( \frac{1}{N} \sum_{n=1}^N \Cap(x_n, \Prefix) \right)^{-1} b(\Prefix, \x, \y)
= \frac{b(\Prefix, \x, \y)}{\Supp(\Prefix, \x)},
\end{align*}
where for the second equality, we used the definitions of $\NCap$~\eqref{eq:num-cap}
and $\Prefix$'s lower bound~\eqref{eq:lower-bound}, and for the last equality,
we used the definition of $\Prefix$'s normalized support~\eqref{eq:support}.

The curiosity for a prefix~$\Prefix$ is thus also equal to its objective lower bound,
scaled by the inverse of its normalized support.
%
For two prefixes with the same lower bound, curiosity gives higher priority to
the one that captures more data.
%
This is a well-motivated scheduling strategy if we model prefixes that extend
the prefix with smaller support as having more `potential' to make mistakes.
%
We note that using curiosity in practice does not introduce new bit vector
or other expensive computations; during execution, we can calculate curiosity
as a simple function of already derived quantities.

In preliminary experiments, we observe that using a priority queue ordered by
curiosity sometimes yields a dramatic reduction in execution time,
compared to using a priority queue ordered by the objective lower bound.
%
Thus far, we have observed significant benefits on specific small problems,
where the structure of the solutions happen to render curiosity particularly
effective (not shown).
%
Designing and studying other `curious' functions, that are effective in more
general settings, is an exciting direction for future work.
