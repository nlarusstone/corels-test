\section{Experiments}
\label{sec:experiments}

\begin{figure}[t!]
\begin{center}
% left lower right upper
\includegraphics[trim={10mm, 11mm, 25mm, 5mm},
width=0.48\textwidth]{figs/compare-compas-weapon.pdf}
\end{center}
\caption{Test accuracy means (white squares),
standard deviations (error bars),
and values (colors correspond to folds).
}
\label{fig:comparison}
\end{figure}

\begin{figure}[t!]
\begin{center}
% left lower right upper
\includegraphics[trim={12mm, 0mm, 24mm, 5mm}, width=0.43\textwidth]{figs/compas-sparsity-training.pdf}
\includegraphics[trim={12mm, 12mm, 24mm, 1mm}, width=0.43\textwidth]{figs/frisk-sparsity-training.pdf}
\end{center}
\caption{Training and test accuracy as a function of model size.
%
For CORELS, CART, and C4.5, we vary the regularization parameter~$\Reg$
and complexity parameters~$cp$ and~$C$, respectively.
% and are indicated within parentheses in the legend.
%
Legend markers and error bars indicate means and standard deviations,
respectively, of test accuracy across cross-validation folds.
%
Small circles mark associated training accuracy means.
%
Top:  %Two-year recidivism prediction for the ProPublica COMPAS dataset.
ProPublica dataset.
%
None of the models exhibit significant overfitting.
%mean training accuracy never exceeds mean test accuracy
%by more than about 0.01.
%
Bottom:  %Weapon prediction for the NYCLU stop-and-frisk dataset.
NYCLU dataset.
%
Only CART with ${cp = 0.001}$ significantly overfits.
%
We do not depict C4.5, which finds large models (${>100}$ leaves)
and dramatically overfits for all tested parameters.
}
\label{fig:sparsity}
\end{figure}

% CART implementation in R's rpart
% C4.5 = J48 in RWeka
% RIPPER in R's caret

Our experimental analysis addresses four questions:
(1) How does CORELS' accuracy and model size compare to other algorithms'?
(2) How rapidly does the objective function converge?
(3) How rapidly does CORELS prune the search space?
(4) How much does each of the implementation optimizations contribute to CORELS' performance?
%
All timed results were executed on a server with two Intel Xeon E5-2699~v4
(55~MB cache, 2.20~GHz) processors and 448~GB RAM.
%
Except where we mention a memory constraint, all experiments
can run comfortably on smaller machines, \eg a laptop with 16GB~RAM.
%
We focus on two problems:
using the ProPublica dataset~\citep{LarsonMaKiAn16} to predict two-year recidivism
and using the NYCLU 2014 stop-and-frisk dataset~\citep{nyclu:2014} to predict
whether a weapon will be found on a stopped individual who is frisked or searched.~\footnote{We
present additional empirical results, and further implementation and data processing details
in a long version of this report~\citep{AngelinoLaAlSeRu17}.}

We first ran a 10-fold cross validation experiment using CORELS
and eight other algorithms:~\footnote{We use standard R packages, with default
parameter settings, for the first seven.}
logistic regression, support vector machines, AdaBoost, CART, C4.5, random forests,
RIPPER,~\footnote{We were unable to execute RIPPER for the NYCLU problem.}
and scalable Bayesian rule lists (SBRL).~\footnote{Code for SBRL can be found at
\url{https://github.com/Hongyuy/sbrlmod}.}
%
Figure~\ref{fig:rule-list} shows an  optimal rule list that CORELS learns
for the ProPublica dataset.
%
Figure~\ref{fig:comparison} shows that there were no statistically significant
differences in algorithm accuracies.
\begin{arxiv}
In fact, the difference between folds was far larger than the difference
between algorithms.
We conclude that CORELS produces models whose accuracy is comparable
to those found via other algorithms.

\end{arxiv}
%
Figure~\ref{fig:sparsity} summarizes differences in accuracy and model size
for CORELS and other tree (CART, C4.5) and rule list (RIPPER, SBRL) learning algorithms.
%
For both problems, CORELS can learn short rule lists without sacrificing accuracy.

\begin{figure}[t!]
\begin{center}
% left lower right upper
\includegraphics[trim={20mm, 25mm, 24mm, 15mm}, width=0.45\textwidth]{figs/compas_execution-remaining-space.pdf}
\end{center}
\caption{CORELS with (lines) and without
(dashes) the equivalent points bound (Theorem~\ref{thm:identical}).
%
%Horizontal axes plot wall clock time (log scale).
%
Top: Objective value (thin line) and lower bound (thick line) for CORELS,
and lower bound (dashes) for an execution without the equivalent points bound.
%as a function of wall clock time (log scale).
%
Numbered points
%along the objective
indicate when the length of the best known rule list changes,
and are labeled by the new length.
%
%CORELS quickly the optimal value (star marker),
%and certifies optimality when the lower bound matches the objective value.
A star marks the optimum.
%
% A separate execution of
%CORELS without the equivalent points bound remains far from complete,
%and its lower bound (dashed line) far from the optimum.
%
Bottom: $\lfloor \log_{10} \Remaining(\CurrentObj, \Queue) \rfloor$,
%as a function of wall clock time,
where~$\Remaining(\CurrentObj, \Queue)$
is the upper bound on remaining search space size
(Theorem~\ref{thm:remaining-eval-fine}).
}
\label{fig:objective}
\end{figure}

\begin{table}[t!]
\centering
\resizebox{0.49\textwidth}{!}{
\begin{tabular}{l | c | c | c | c | c}
 & $t_\text{total}$ & $t_\text{opt}$ & $i_\text{total}$ & $Q_\text{max}$ & $K_\text{max}$ \\
Algorithm variant & (min) & (s) & ($\times 10^6$) & ($\times 10^6$) & \\
\hline
CORELS & 5.5 (1.6) & 8 (2) & 1.7 (0.4) & 1.3 (0.4) & 5-6 \\
No priority queue (BFS) & 6.7 (2.2) & 4 (1) & 1.9 (0.6) & 1.5 (0.5) & 5-6 \\
No support bounds & 10.2 (3.4) & 13 (4) & 2.7 (0.8) & 2.2 (0.7) & 5-6 \\
No symmetry-aware map & 58.6 (23.3) & 23 (6) & 16.0 (5.9) & 14.5 (5.7) & 5-6 \\
No lookahead bound & 71.9 (23.0) & 9 (2) & 18.5 (5.9) & 16.3 (5.3) & 6-7 \\
%equiv. pts. bound & 188.6 (104.6) & 6178 (1840) & 803.8 (0.1) & 790.5 (0.4) & 10-10
No equivalent pts bound & $>$134 & $>$7168 & $>$800 & $>$789 & $\ge$10
\end{tabular}
}
\vspace{4mm}
\caption{Per-component performance improvement.
%
The columns report total execution time,
time to optimum, number of queue insertions,
maximum queue size, and maximum evaluated prefix length.
%
The first row shows CORELS; subsequent rows show variants
that each remove a specific implementation optimization or bound.
%
We terminated each experiment in the last row after consuming 390-410GB RAM.
%
In all but the final row and column, we report means
(and standard deviations) over 10 cross-validation folds;
in the final row, we report the minimum values across folds.
}
\label{tab:ablation}
\vspace{-8mm}
\end{table}

In the remainder, we show results using the ProPublica dataset.
%
Figure~\ref{fig:objective} illustrates how both the objective and the size of
the remaining search space decrease as CORELS executes.
The objective drops quickly, achieving the optimal value within 10 seconds.
CORELS certifies optimality in less than 6 minutes --
the objective lower bound of the remaining search space
steadily converges to the optimal objective as the search space shrinks.

Finally, we determine the efficacy of each of our bounds and data structure optimizations.
%
Figure~\ref{fig:objective} highlights how a separate execution of CORELS without
the equivalent points bound remains far from complete,
with its lower bound far from the optimum.
%
Table~\ref{tab:ablation} provides summary statistics for experiments using
the full CORELS implementation and variants that each remove a specific optimization.
%
Figure~\ref{fig:queue} presents a view of the same experiments, focusing
on three of our optimizations. These plots depict the number of
prefixes of a given length in the queue during the algorithm's execution.

\begin{figure}[t!]
\begin{center}
% left lower right upper
\includegraphics[trim={30mm 15mm 35mm 30mm},
width=0.45\textwidth]{figs/kdd_compas_ablation_small-queue.pdf}
\end{center}
\caption{Queue composition.
%
Numbers of prefixes in the queue (log scale), labeled and colored by length,
as a function of wall clock time (log scale), for CORELS (top left),
and without three specific implementation optimizations or bounds.
%
The gray shading fills in the area beneath the total number of
queue elements for CORELS.
\ie the sum over all lengths in the top left figure.
%
For comparison, we replicate the same gray region
in the other three subfigures.
}
\label{fig:queue}
\end{figure}
