\section{Preliminary experiments}
\label{sec:experiments}

\begin{figure}[t!]
\begin{center}
%\includegraphics[width=0.75\textwidth]{figs/sketch-comparison.png}
\begin{arxiv}
\includegraphics[width=0.75\textwidth]{figs/compare-compas.pdf}
\end{arxiv}
\begin{kdd}
\includegraphics[width=0.5\textwidth]{figs/compare-compas.pdf}
\end{kdd}
\end{center}
\caption{Comparison with other methods:
Test error for us and a few other algorithms
(CART, C4.5, CBA, CMAR/CPAR, C5.0, Ripper, \dots),
as a function of sparsity over 10 folds, for one big dataset (box plots).
Also report algorithm runtimes (mean $\pm$ standard deviation over 10 folds).}
\label{fig:comparison}
\end{figure}

\begin{arxiv}
\begin{figure}[t!]
\begin{center}
\end{center}
\caption{Missing:  Test error as a function of regularization and sparsity
(number of rules) as a function of regularization, over 10 folds,
for one big dataset.}
\label{fig:regularization}
\end{figure}
\end{arxiv}

\begin{arxiv}
\begin{figure}[t!]
\begin{center}
\includegraphics[width=0.75\textwidth]{figs/sketch-ablation.png}
\end{center}
\caption{Ablation experiment:
Show the effect of each ``piece'' at a time,
run X without each in turn and show the difference in either
quality of solution or runtime or amount of memory, size of cache or queue,
where X is a specific implementation
(meaning a specific scheduling policy and node type)}
\label{fig:ablation}
\end{figure}
\end{arxiv}

\begin{arxiv}
\begin{figure}[t!]
\begin{center}
\end{center}
\caption{Missing:  Some sort of comparison of different scheduling policies}
\label{fig:scheduling-policy}
\end{figure}
\end{arxiv}

\begin{arxiv}
\begin{figure}[t!]
\begin{center}
%\includegraphics[width=0.65\textwidth]{figs/sketch-queue-size.png}
\includegraphics[width=0.8\textwidth]{figs/ela_compas-queue-cache-size-insertions.pdf}
\end{center}
\caption{Cache and queue data structure sizes and insertions.
%
The top plot shows the sizes of the cache and queue data structures,
as a function of wall clock time.
%
The number of nodes in the cache (solid black line) is an
upper bound on the number of elements in the physical queue
(dotted gray line), since the physical queue elements only
correspond to the cache trie data structure's leaf nodes
plus disconnected cache nodes that have been marked for deletion.
%
The queue's physical size is an upper bound on its
logical size (solid blue line), which doesn't include nodes
that have been marked for deletion.
%
The bottom plot shows the cumulative number of cache insertions,
which is equivalent to the cumulative number of queue insertions,
as a function of wall clock time.
}
\label{fig:queue-cache-size-insertions}
\end{figure}
\end{arxiv}

\begin{figure}[t!]
\begin{center}
%\includegraphics[width=0.65\textwidth]{figs/sketch-objective.png}
\begin{arxiv}
\includegraphics[width=0.8\textwidth]{figs/ela_compas-remaining-space.pdf}
\end{arxiv}
\begin{kdd}
\includegraphics[width=0.5\textwidth]{figs/ela_compas-remaining-space.pdf}
\end{kdd}
\end{center}
\caption{Top: Objective value and lower bound as a function of wall clock time.
%
Missing: horizontal lines and x-ticks for CART and C4.5.
%
Bottom:
The logarithm (base 10) of the upper bound
from Proposition~\ref{prop:remaining-eval-coarse}
on the size of the remaining search space,
as a function of wall clock time.
%
This bound depends on the total number of available rules,
as well as two dynamic quantities:
the current best objective value,
and the histogram of logical queue elements,
partitioned by prefix length.}
\label{fig:objective}
\end{figure}

\begin{figure}[t!]
\begin{center}
\begin{arxiv}
\includegraphics[width=0.8\textwidth]{figs/ela_compas-queue.pdf}
\end{arxiv}
\begin{kdd}
\includegraphics[width=0.5\textwidth]{figs/ela_compas-queue.pdf}
\end{kdd}
\end{center}
\caption{Logical queue composition.
}
\label{fig:queue}
\end{figure}

%\begin{figure}[t!]
%\begin{center}
%%\includegraphics[width=0.65\textwidth]{figs/sketch-search-space.png}
%\begin{arxiv}
%\includegraphics[width=0.8\textwidth]{figs/ela_compas-remaining-space.pdf}
%\end{arxiv}
%\begin{kdd}
%\includegraphics[width=0.5\textwidth]{figs/ela_compas-remaining-space.pdf}
%\end{kdd}
%\end{center}
%\caption{The logarithm (base 10) of the upper bound
%from Proposition~\ref{prop:remaining-eval-coarse}
%on the size of the remaining search space,
%as a function of wall clock time.
%%
%This bound depends on the total number of available rules,
%as well as two dynamic quantities:
%the current best objective value,
%and the histogram of logical queue elements,
%partitioned by prefix length.
%}
%\label{fig:search-space}
%\end{figure}

\begin{arxiv}
\begin{figure}[t!]
\begin{center}
%\includegraphics[width=0.65\textwidth]{figs/sketch-max-length.png}
\includegraphics[width=0.8\textwidth]{figs/ela-max-length-check.png}
\end{center}
\caption{Max prefix length over time (computed from objective value)}
\label{fig:max-length}
\end{figure}

\begin{figure}[t!]
\begin{center}
\includegraphics[width=0.8\textwidth]{figs/ela_compas-prefix-length.pdf}
\end{center}
\caption{Best prefix length over time}
\label{fig:prefix-length}
\end{figure}
\end{arxiv}
